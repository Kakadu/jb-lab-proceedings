\section{Введение}
С каждым годом сложность программных продуктов возрастает, и, соответственно, задача обеспечения их надёжности и корректности становится всё более востребованной. Для её решения используются активно развивающиеся техники из области верификации. Одной из таких техник является символьное исполнение~\cite{wenenhancing, li2014symbolic}.

Некоторые современные верификаторы~\cite{sethu2018systems, wang2017dependence, sharma2018veritesting} используют технику символьного исполнения для сведения задачи анализа программного кода с неопределёнными входными данными к проверке выполнимости формул логики первого порядка. Для такой проверки чаще всего используются SMT-решатели (Satisfiabilty Modulo Theories). Данные инструменты позволяют определить, выполнима ли формула, записанная в поддерживаемых
решателями теориях (например, теория линейной арифметики, битовых векторов и некоторые другие). Таким образом в процессе сведения анализа программного кода к проверке выполнимости формул логики необходимо моделировать семантику программы в теориях логики первого порядка, поддерживаемых решателями.

В современных языках программирования, таких как \clang{}, \textsc{C++}, \textsc{C\#}, \textsc{Java} и других, память программы достаточно полно отражает состояние этой программы, и операции с памятью имеются почти в каждой нетривиальной программе. Соответственно, разработка метода моделирования операций с памятью для последующего использования в SMT-решателе является важной и актуальной задачей верификации. Метод моделирования операций с памятью (т. е. \emph{модель памяти}) определим, согласно статье~\cite{mandrik}, как формальное представление указателя и ссылки, а также формализацию результата операций над ними c использованием логических формул. Разработка такого метода позволила бы доказывать, что в программе, например, отсутствуют: утечки памяти, некорректные доступы к памяти (например, выход за границы массива), неопределённые поведения и т.д. Однако задача разработки модели памяти является нетривиальной, а её решение зависит от семантики конкретного языка программирования.

Важной особенностью платформы .NET является возможность использования динамической памяти, т.е. памяти заранее не ограниченного размера. Помимо этого в платформе .NET существует два контекста исполнения кода: безопасный (англ. safe) и небезопасный (англ. unsafe). В случае безопасного контекста, низкоуровневыми операциями с памятью управляет \textsc{CLR} (Common Language Runtime). Когда контекст исполнения является небезопасным, пользователю становятся доступны операции с указателями, как в языке \clang{}, которые могут привести к \emph{реинтерпретации данных}.

Под реинтерпретацией понимается рассмотрение данных при помощи структуры\footnote{в широком смысле, от слова <<структурированный>>}, отличной от той, которой они изначально задавались. Такого поведения можно добиться с помощью адресной арифметики и приведения типов указателей, например, прочитав массив из элементов типа \lstinline[language=csharp]{int} указателем с типом \lstinline[language=csharp]{byte*}.

В данной статье представлен метод моделирования операций с памятью платформы .NET в условиях безопасного и небезопасного контекста. Для его создания был проведён обзор существующих на данный момент моделей памяти.
