\section*{Введение}
Ошибки в программном обеспечении оказывают прямое воздействие на его востребованность. 
В связи с этим приобрели популярность средства статического и динамического анализа кода, 
которые помогают разработчикам на ранних этапах находить и устранять ошибки.

Верификация программ является продвинутым инструментом, помогающим разработчикам. Её основное свойство~--- доказательство корректности программы. То есть, 
разработчик получает артефакт, утверждающий, что программа соответствует спецификации.

Одним из способов проведения верификации является статическое символьное исполнение в сочетании с проверкой моделей \foreignlanguage{english}{(model checking)}.
Идея символьного исполнения~\cite{king1976symbolic} заключается в исполнении кода не над конкретными значениями, а над символьными выражениями, каждое из которых представляет типизированное множество значений, которые может принять аргумент или локальная переменная.

В отличие от стандартного исполнения кода, символьное --- происходит не по определенному пути,
а сразу над множеством путей одновременно. При этом для каждого пути можно задать такой конкретный набор значений параметров, что реальное исполнение пройдет тот же путь. Проверка моделей используется для анализа состояний программы, полученных с помощью символьного исполнения, на отсутствие ошибок.

Для того чтобы доказать корректность программы необходимо каким-то образом получить все пути исполнения программы и для каждого пути доказать, что он не приводит к ошибке. Это сразу ставит проблему взрыва множества путей исполнения программы~\cite{baldoni2018survey}, 
поскольку в общем случае программа может содержать инструкции ветвления. 
Однако более существенна другая проблема~--- потенциально бесконечные пути исполнения. Она проявляется, если в программе есть циклы, зависящие от символьных переменных.

Существующие символьные виртуальные машины либо раскручивают отношение перехода программы~\cite{mcmillan2010lazy}, либо пытаются выводить индуктивный инвариант цикла~\cite{jaffar2012tracer}. Минусом первого подхода является приближение снизу множества достижимых состояний программы, что препятсвует верификации программы. 
Второй подход может зацикливаться на некоторых примерах.

В данной работе предлагается новый способ преодоления представленных проблем, основанный на \emph{исчислении композициональных символьных куч} и композициональном символьном исполнении~\cite{godefroid2007compositional}.
Исчисление вводит формализм \emph{символов куч}, с помощью которого можно описывать состояния после исполнения циклического региона программы.

Результатами данной работы являются алгоритм композиционального символьного исполнения без раскрутки отношения перехода программы и интерпретатор промежуточного языка CIL платформы .NET, которые были
реализованы в проекте V\#, виртуальной символьной виртуальной машине для анализа .NET. Была проведена апробация  алгоритма и интерпретататора на библиотеках \foreignlanguage{english}{VSharp.Test, Chess.Net}.
