\section{Алгоритм композиционального символьного исполнения}

% Легко заметить, что рекурсивные символы и их описания, порождаемые приведённым методом, соответствуют обобщённым кучам и уравнениям на состояния. 
Благодаря соответствию между рекурсивными символами и их описаниями, с одной стороны, и обобщёнными кучами $\GRec{id}$ и $\body{id}$, с другой стороны, можно 
получить \emph{алгоритм автоматической проверки достижимости ошибок в программах с произвольными графами потока управления}, не раскручивающий отношение перехода. Для алгоритма важно понятие \emph{состояния исполнения} и уникального идентификатора $id$.

\begin{defn}
\emph{Состояние исполнения}~--- это кортеж $(l,pc,\sigma,D)$, где $l$~--- номер инструкции,
$pc$~--- \emph{условие пути} (path condition~--- символьная формула, описывающая ограничения на достижимость инструкции), $\sigma$~--- символьная (возможно, обобщённая) куча, являющаяся отображением из локаций в символьные выражения, $D$~--- множество посещённых рекурсивных вершин.

\emph{Уникальный идентификатор $id$}~--- это $(l,D)$, где $l$~--- номер инструкции, которая является рекурсивной вершиной в CFG, а $D$~--- множество посещенных рекурсивных вершин.
\end{defn}

Новый алгоритм описан в стиле алгоритма классического символьного исполнения (см. раздел~\nameref{section:review}). Для того чтобы продемонстрировать разницу между последним, новые части алгоритма представлены на зеленом фоне. Необходимые для алгоритма множества состояний $Body(\cdot,\cdot)$, описывающие поведения циклических фрагментов графа потока управления, и множество $RV$ \emph{рекурсивных вершин}, определенных в предыдущем разделе, объявлены глобально. Аналогично, чтобы не загромождать описание алгоритмов, посчитаем, что функция $\sim$ определения эквивалентности двух отображений (символьных куч) определена глобально.

Основу алгоритма представляет функция $Exec$, которая может быть вызвана либо от \emph{начальной} \smartref{compositional_algo}{alg:start_vertex}, 
либо от \emph{рекурсивной} вершины (строка~\ref{alg:start_recursive_state}). В первом случае её результатом является  множество путей, приводящих к ошибкам (строка~\ref{alg:print_errors}).
Во втором случае результатом является множество символьных куч $Body(l_0,D_0)$, описывающее поведение циклического участка графа потока управления:
в строках~\ref{alg:deal_with_new_state_start}~--~\ref{alg:add_recursive_state} 
происходит добавление \emph{обобщённого состояния}, представляющего собой композицию \emph{построенного состояния} $\sigma'$ с \emph{рекурсивным состоянием} $Rec(l_0,D_0)$, 
а в конце функции (строка~\ref{alg:epsilon_equation_start}) добавляется завершающее состояние для случая, 
когда исполнение не вернулось в \emph{рекурсивную} вершину $l_0$,
соответствующее \emph{пустому пути} $\varepsilon$ из определения рекурсивных символов в методе описания путей в графе.


Согласно схеме алгоритм выбирает следующее состояние исполнения из рабочего множества ($pickNext$), 
исполняет соответствующую инструкцию, порождая новые состояния исполнения 
(строки~\ref{alg:execute_instruction_start}~--~\ref{alg:execute_instruction_end}), и добавляет их в рабочее множество, 
объединяя в одно те состояния исполнения, у которых равны номера инструкций $l$ и множества \emph{посещенных} рекурсивных вершин $D$
(строки~\ref{alg:merge_start}~--~\ref{alg:merge_end}).
Стоит заметить, что предлагаемый алгоритм не раскручивает циклы, 
а вводит \emph{символы рекурсивных куч} $Rec(l_0, D_0)$ (строка~\ref{alg:enter_recursive_symbol})
и множества обобщённых куч $Body(l_0,D_0)$ (строка~\ref{alg:start_recursive_state}) для их описания.
Все $Body(\cdot,\cdot)$ используются для определения выполнимости ограничений пути 
(строки~\ref{alg:sat1},~\ref{alg:sat2},~\ref{alg:sat3}).

\begin{algorithm}
    \caption{Алгоритм композиционального символьного исполнения}\label{compositional_algo}
\begin{algorithmic}[1]
    \State $\forall l \in RV, \forall D \quad Body(l,D) \gets \emptyset$
    \State \Return $\text{$\textsc{Exec}(start,\emptyset)$}$ \label{alg:start_vertex}
    
    \Procedure{\textsc{Exec}}{($l_0$ : Vertex, $D_0$ : Vertex set)}
        \State{$pc_r, \sigma_r \gets (\bot,\emptyheap)$}
        \State{$W \gets \{(l_0,\top,\emptyheap,D_0)\}$ $Errors \gets \emptyset$}
        \While{$W\neq\emptyset$} \label{alg:while_start}
            \State $(l,pc,\sigma,D), W \gets pickNext(W)$
            \State $S \gets \textsc{ExecuteInstruction}(l,pc,s)$
            % \State \Comment{Объединим состояния с эквивалентными из множества $W$}
            \For{$(l',pc',\sigma') \in S$}
                \State $W \gets \textsc{HandleState}(W, l',pc',s', D, l_0, D_0)$
            \EndFor
        \EndWhile \label{alg:while_end}
    
        \tikzmk{A}
        \If{$l_0 \in RV$}
            \State $guard_{\emptyheap} \gets \top$ \label{alg:epsilon_equation_start}
            \For{$(\cdot,pc,\cdot,\cdot) \in Body(l_0,D_0)$}
               \State $guard_{\emptyheap} \gets guard_{\emptyheap} \land \lnot pc$
            \EndFor

            \State $Body(l_0,D_0) \gets \textsc{Join}(Body(l_0,D_0), l_0, guard_{\emptyheap}, \emptyheap, D_0)$ \label{alg:epsilon_equation_end}
        \Else
            \tikzmk{B}
            \boxit{green}
                \text{\texttt{print} $Errors$}\label{alg:print_errors}\;  
        \EndIf
    \EndProcedure
\end{algorithmic}
\end{algorithm}

Приведем дальше краткое описание вспомогательных функций.
Аналогично разделу~\nameref{section:review} функция \textsc{ExecuteInstruction} символьно исполняет инструкции демонстрационного языка. Единственное изменение~--- это новая реализация (которая не приводится в этой статье) функции $SAT$ для определения выполнимости ограничений: трансляция полученных множеств состояний для циклических участков графа $Body(\cdot,\cdot)$ в дизъюнкты Хорна и непосредственное решение этих ограничений с помощью Хорн-решателя.

\begin{algorithm}
    \caption{Модифицированная функция \textsc{ExecuteInstruction}} \label{new_execute_instruction}
\begin{algorithmic}[1]
    \Procedure{\textsc{ExecuteInstruction}}{$l_0$, $pc$, $\sigma$}
        \State $S \gets \emptyset$
        \Switch{$instr(l)$} \label{alg:execute_instruction_start}
            \Case{\textsc{\texttt{$v$ := $e$}}\qquad \texttt{// присваивание}}
                \State $S \gets \{(succ(l),pc, \GMutate{\sigma}{\emptyheap[v\mapsto e]})\}$
            \EndCase

            \Case{$if(e)$ goto $l'$ \qquad \texttt{// условный переход}}
                \If{$SAT(\mkGreen{Body}, \sigma, pc \land e)$} \label{alg:sat1}
                    \State $S \gets \{(l', pc \land e, \sigma)\}$
                \EndIf

                \If{$SAT(\mkGreen{Body}, \sigma, pc \land \neg e)$} \label{alg:sat2}
                    \State $S \gets S \cup \{(succ(l), pc \land \neg e, \sigma)\}$
                \EndIf
            \EndCase

            \Case{$assert(e)$ \qquad \texttt{// проверка}}
                \If{$SAT(\mkGreen{Body}, \sigma, pc \land \neg e)$} \label{alg:sat3}
                    \State $Errors \gets Errors \cup \{(\sigma, pc \land \neg e)\}$
                \Else
                    \State $S \gets \{(succ(l),pc,\sigma)\}$
                \EndIf
            \EndCase

            \Case{halt\qquad \texttt{// завершение программы}}
            \EndCase
        \EndSwitch \label{alg:execute_instruction_end}
        \State \Return $S$
    \EndProcedure
\end{algorithmic}
\end{algorithm}

Новая функция \textsc{HandleState} анализирует полученное состояние и решает, что с ним дальше делать.
\begin{itemize}
    \item Добавить состояние, символьная куча которого представляет собой обобщенную кучу $\sigma' \circ Rec(l_0,D_0)$, в множество состояний для описания циклического фрагмента (строка~\ref{alg:add_recursive_state}).
    \item Забыть состояние, если перешли в уже посещенную \emph{рекурсивную} вершину (строка~\ref{alg:forget_state}).
    \item Вызвать функцию \textsc{Exec} для построения тела описания рекурсивного состояния, если перешли в новую \emph{рекурсивную} вершину (строка~\ref{alg:start_recursive_state}).
    \item Добавить состояние в множество рабочих состояний для исследования (строка~\ref{alg:call_join_from_handle}).
\end{itemize}
Её результатом является новое множество $W$.

\begin{algorithm}
    \caption{Новая функция \textsc{HandleState}} \label{new_handle_state}
\begin{algorithmic}[1]
    \Procedure{\textsc{HandleState}}{$W$, $l'$, $pc'$, $\sigma'$, $D$, $l_0$, $D_0$}
        \If{$l' = l_0$} \label{alg:deal_with_new_state_start}
            \State $\sigma' \gets \sigma' \circ Rec(l_0,D_0)$
            \State $Body(l_0,D_0) \gets \text{\textsc{Join}($Body(l_0,D_0), l_0, pc',\sigma',D$)}$
            \State \Return W \label{alg:add_recursive_state}
        \ElsIf{$l' \in D$} \label{alg:forget_state}
            \State \Return
        \ElsIf{$l' \in RV$}
            \State $D \gets D \cup \{l'\}$
            \State $\sigma' \gets \sigma' \circ Rec(l',D)$ \label{alg:enter_recursive_symbol}
            \If{$Body(l',D) = \emptyset$}
                \State $Body(l',D) \gets \text{\textsc{Exec}}(l',D)$ \label{alg:start_recursive_state}
            \EndIf
        \EndIf

        \State \Return \text{$\textsc{Join}(W, l', pc', \sigma', D)$} \label{alg:call_join_from_handle}\;
    \EndProcedure
\end{algorithmic}
\end{algorithm}

Как и в разделе~\nameref{section:review}, функция \textsc{Join} пытается найти эквивалентное состояние для объединения из рабочего множества $W$. Поскольку сейчас состояние~--- это четверка, то необходимо учитывать множество посещенных рекурсивных вершин $D$, которое влияет на отношение перехода анализируемой программы.
Очевидно, что можно объединять только те состояния, у которых равны представленные множества.

\begin{algorithm}
    \caption{Модифицированная функция \textsc{Join}} \label{new_join}
\begin{algorithmic}[1]
    \Procedure{\textsc{Join}}{$W$, $l$, $pc$, $\sigma$, $D$}
        \If{$\exists (l', pc', \sigma', D') \in W : l = l'$ \mkGreen{$\land D = D'$}} \label{alg:merge_start}
            \State $W \gets W \setminus \{(l',pc',\sigma',\mkGreen{$D'$})\}$
            \State $W \gets W \cup \{(l,pc \lor pc',\Merge{\pair{pc}{\sigma},\pair{pc'}{\sigma'}},\mkGreen{D})\}$
        \Else
            \State $W \gets W \cup \{(l, pc, \sigma, \mkGreen{D})\}$ \label{alg:merge_end}
        \EndIf

        \State \Return W
    \EndProcedure
\end{algorithmic}
\end{algorithm}


