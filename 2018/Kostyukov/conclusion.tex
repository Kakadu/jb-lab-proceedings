\section{Заключение}

В данной работе описан подход к композициональному символьному исполнению без раскрутки. Была предложена концепция композициональной памяти с символьной адресацией. Был доказан некоторый набор свойств КСП, дающий основание для подхода в стиле систем переписывания, где символьные кучи могут сами выступать как символы. Это даёт возможность автоматически порождать уравнения над состояниями, решения которых в точности отражают поведения функций, работающих с динамической памятью. Было показано как свести задачу решения уравнений над состояниями к задаче проверки безопасности чистых функций второго порядка.

Данная работа нацелена на теоретические основания композиционального анализа динамической памяти. Мы оставляем апробацию этого подхода на будущее. Другим направлением будущих исследований может быть расширение нашего формализма на композициональный анализ параллельных программ.