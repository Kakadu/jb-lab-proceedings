\section{Символьное исполнение}
\label{sec:traditional-se}

\newcommand{\traditionalalgo}
{
\begin{algorithm2e}
	\scriptsize
	\SetKw{Yield}{yield}
	\SetKw{Break}{break}
	\SetKwProg{Fn}{Function}{}{}
	\SetKwIF{If}{ElseIf}{Else}{if}{then}{elif}{else}{}
	\SetKwSwitch{Switch}{Case}{Other}{switch}{}{case}{otherwise}{}{}
	\SetKwFor{For}{foreach}{do}{}
	\SetAlgoSkip{}
		\KwIn{Корректно типизированная программа $P$, написанная на языке из~\autoref{sec:language}, исследуемая функция $F$}
	\KwOut{Символьное куча $F$, множество условий путей ошибочных трасс $Errors$}
	\BlankLine
		\assign{$\sigma, Errors$}{\texttt{initSymbolicState}(\emph{F}), $\emptyset$\label{algo:line:initState}}\;
		\Return{ExecStatement\textsc{(}F.body, $\sigma$, true\textsc{)}}\;
	\BlankLine
	\Fn{ExecStatement(stmt : Statement, $\sigma$ : SymbolicHeap, $\pi$ : PathCondition) : SymbolicHeap} {
		\Switch{stmt}{
			\lCase{\underline{\textsc{nop}}\,\textbf{:}}{
				\Return{$\sigma$}
			}
			\uCase{\underline{\textsc{\{}} statements \underline{\textsc{\}}}\,\textbf{:}}{
				\For{childStmt $\in$ statements}{
					\assign{\emph{$\sigma$}}{\emph{ExecStatement}(\emph{childStmt}, \emph{$\sigma$}, \emph{$\pi$})}\;
				}
				\Return{$\sigma$}\;
			}
			\uCase{\underline{\textsc{if}} cond \underline{\textsc{then}} thenBranch \underline{\textsc{else}} elseBranch\,\textbf{:}}{
				\assign{\emph{c}}{\emph{Eval}(\emph{cond}, $\sigma$)}\;
				\Return{Branch\textsc{(}c, $\sigma$, $\pi$,\\
					\qquad$\lambda\,\pi'.$ExecStatement\textsc{(}thenBranch, fork\textsc{(}$\sigma$\textsc{)}, $\pi'$\textsc{)},\\
					\qquad$\lambda\,\pi'.$ExecStatement\textsc{(}elseBranch, fork\textsc{(}$\sigma$\textsc{)}, $\pi'$\textsc{))}}\;
			}
			\uCase{\underline{\textsc{fail}}\,\textbf{:}}{
				\assign{$Errors$}{$Errors \cup \{\pi\}$}\;\label{algo:line:abort1}
				\Return{$\sigma$}\;
			}
			\uCase{x $:=$ \underline{\textsc{alloc}}\textsc{(}\emph{$\text{e}_1$, \ldots, $\text{e}_\text{n}$}\textsc{)}\label{algo:line:heap-manipulation-begin}\,\textbf{:}}{
				\assign{(\emph{$\sigma'$}, \emph{freshRef})}{\texttt{allocateInHeap}\big(\emph{$\sigma$}, ($\text{\emph{Eval}(e}_1, \sigma\text{)},\ldots,\text{\emph{Eval}(}\text{e}_\text{n}, \sigma\text{)}$)\big)\label{algo:line:alloc}}\;
				\Return{\texttt{\textsc{writeHeap}}\textsc{\big(}$\sigma'$, x, freshRef\textsc{\big)}\label{algo:line:write-in-alloc}}
			}
			\lCase{x $:=$ expr\,\textbf{:}}{
				\Return{\texttt{\textsc{writeHeap}}\textsc{\big(}$\sigma$, x, Eval\textsc{(}expr, $\sigma$\textsc{)}\textsc{\big)}\label{algo:line:write-in-assign}}
			}
			\uCase{\textsc{(}$\text{x}_\text{1}$, \ldots, $\text{x}_\text{n}$\textsc{)} $:=$ \underline{\textsc{read}}\textsc{(}reference\textsc{)}\,\textbf{:}}{
				\assign{\emph{r}}{\emph{Eval}(\emph{reference}, $\sigma$)}\;
				\Return{Branch\textsc{\Big(}r $=$ null, $\sigma$, $\pi$,\\
					\qquad$\lambda\,\pi'.$ExecStatement\textsc{(}\underline{\textsc{fail}}, $\sigma$, $\pi'$\textsc{)}\label{algo:line:abort2},\\
					\qquad$\lambda\,\pi'.$\texttt{\textsc{writeHeap}}\textsc{\big(}$\sigma$, $\text{x}_\text{1}$, \ldots, $\text{x}_\text{n}$, \emph{\texttt{readHeap}(\emph{$\sigma$}, \emph{r})}\textsc{\big)\textsc{\Big)}\label{algo:line:read}}\label{algo:line:write-in-read}}\;
			}
			\uCase{\underline{\textsc{write}}\textsc{(}reference, \emph{$\text{e}_1$, \ldots, $\text{e}_\text{n}$}\textsc{)}\,\textbf{:}}{
				\assign{\emph{r}}{\emph{Eval}(\emph{reference}, \emph{state})}\;
				\Return{Branch\textsc{\Big(}r $=$ null, $\sigma$, $\pi$,\\
					\qquad$\lambda\,\pi'.$ExecStatement\textsc{(}\underline{\textsc{fail}}, $\sigma$, $\pi'$\textsc{)}\label{algo:line:abort3},\\
					\qquad$\lambda\,\pi'.$\texttt{\textsc{writeHeap}}\textsc{\big(}$\sigma$, r, \emph{(}$\text{Eval}\emph{(}\emph{e}_1, \sigma\emph{)},\ldots,\text{Eval}\emph{(}\emph{e}_\emph{n}, \sigma\emph{)}$\emph{)}\textsc{\big)}\textsc{\Big)}\label{algo:line:write}}\;\label{algo:line:heap-manipulation-end}
			}
			\uCase{\underline{\textsc{call}} func\,\textbf{:}}{
				\assign{\emph{f}}{\emph{Eval}(\emph{func})}\;
				\Return{Call\textsc{(}f, $\sigma$, $\pi$\textsc{)}}\;
			}
		}
	}
\caption{Классическое символьное исполнение}
\label{algo:traditional-se}
\end{algorithm2e}
\begin{algorithm2e}
	\scriptsize
	\SetKwProg{Fn}{Function}{}{}
	\SetKwIF{If}{ElseIf}{Else}{if}{then}{elif}{else}{}
	\SetAlgoSkip{}
	\Fn{Branch(c, $\sigma$, $\pi$, then, else) : SymbolicHeap} {
		\lIf{isSatisfiable($\pi\,\,\land$\, c)}{\label{algo:line:then-branch-begin}
			\assign{\emph{$\sigma_{then}$}}{\emph{then}($\pi\,\,\land$\, c)}
		}
		\lElse{\assign{\emph{$\sigma_{then}$}}{\emph{$\sigma$}}}
		\lIf{isSatisfiable($\pi\land\neg$c)}{
			\assign{\emph{$\sigma_{else}$}}{\emph{else}($\pi\land\neg$c)}
		}
		\lElse{\assign{\emph{$\sigma_{else}$}}{\emph{$\sigma$}}}\label{algo:line:else-branch-end}
		\Return{\texttt{\textsc{merge}}($\sigma_{then}$, $\sigma_{else}$)\label{algo:line:merge}}\;
	}
	\Fn{Call(f : Function, $\sigma$ : SymbolicHeap, $\pi$ : PathCondition) : SymbolicHeap\label{algo:line:call-func-begin}} {
		\If{f~--- символьная функция?}{Аппроксимировать применение функции высшего порядка}
		\If{$++$counter\textsc{(}f\textsc{)} $>$ \textsc{BOUND}}{Аппроксимировать неограниченную рекурсию}
		\assign{$\sigma'$}{\emph{ExecStatement}\big(\emph{f.body}, $\sigma$, $\pi$\big)}\;
		$-$$-$\emph{counter}(\emph{f})\;
		\Return{$\sigma'$}\;
	}\label{algo:line:call-func-end}
	\BlankLine
\caption{Классическое символьное исполнение (продолжение)}
\end{algorithm2e}
}

Символьное исполнение~--- это эффективная техника исследования путей исполнения программ. Символьное исполнение отличается от \emph{конкретного} тем, что исполняет программу над \emph{символьными} значениями, которые моделируют множества всех возможных конкретных входных значений. Движки символьного исполнения поддерживают \emph{состояние} программы~--- тройку $(stmt, \sigma, \pi)$, где $stmt$~--- это текущая инструкция, $\sigma$~--- \emph{символьная память} программы (\emph{символьная куча} в нашем случае) и $\pi$~--- \emph{условие пути} текущей ветки исполнения (конъюнкция логических формул~--- предположений, необходимых, чтобы попасть в текущую ветку исполнения).

В символьном исполнении существуют два ключевых подхода к исполнению программ: статическое символьное исполнение (часто используется для верификации)~\cite{koelbl2005constructing,babic2008calysto,dillig2008sound,xie2005scalable} и \emph{динамическое} символьное исполнение (обычно используется для поиска ошибок и порождения тестов)~\cite{tillmann2008pex,godefroid2005dart,bounimova2013billions}. \emph{Динамические} подходы \emph{копируют} состояние каждый раз, когда происходит разветвление; далее исполнение независимо продолжается на копиях начального состояния. Условие пути скопированного состояния будет конъюнкцией начального условия пути и соответствующего условия ветвления. \emph{Статические} подходы производят \emph{объединение состояний} в точках сочленения графа потока управления, что даёт возможность представить всю программу в виде одной логической формулы.

\traditionalalgo

\autoref{algo:traditional-se} описывает обобщённую реализацию движка статического символьного исполнения для демо языка из~\autoref{sec:language}. Эта реализация может быть уточнена для произвольных стратегий поиска и объединения состояний, например~\cite{kuznetsov2012efficient} или~\cite{avgerinos2014enhancing}.

Алгоритм принимает функцию $F$ и сигнализирует о достижении ошибок в коде (строки~\ref{algo:line:abort1},~\ref{algo:line:abort2}~и~\ref{algo:line:abort3}) или возвращает символьное состояние, описывающее то, как функция меняет кучу. Исполнение разветвляется на условных конструкциях (строки~\ref{algo:line:then-branch-begin}"~\ref{algo:line:else-branch-end}) и объединяет получившиеся состояния в точках сочленения графа потока управления (строка~\ref{algo:line:merge}). Выражения вычисляются функцией \emph{Eval} (её реализация опущена). Она преобразует абстрактное синтаксическое дерево выражения в символьную формулу и читает именованные локации.

Строки~\ref{algo:line:heap-manipulation-begin}"~\ref{algo:line:heap-manipulation-end} описывают работу с кучей. Мы формулируем алгоритм в общем виде, скрывая конкретную реализацию функций для работы с кучей: \texttt{initSymbolicState} (строка~\ref{algo:line:initState}), \texttt{readHeap} (строка~\ref{algo:line:read}), \texttt{writeHeap} (строки~\ref{algo:line:write-in-alloc},~\ref{algo:line:write-in-assign},~\ref{algo:line:write-in-read}~и~\ref{algo:line:write}), \texttt{allocateInHeap} (строка~\ref{algo:line:alloc}) и \texttt{merge} (строка~\ref{algo:line:merge}).

Основная проблема масштабируемости символьного исполнения отражена в строках~\ref{algo:line:call-func-begin}"~\ref{algo:line:call-func-end}. Во-первых, алгоритм выполняет \emph{ограниченную раскрутку} рекурсии, что приводит к \emph{\ruquote{взрыву} путей исполнения} (path explosion) в случае динамического символьного исполнения и неограниченному росту размера условий верификации в статическом символьном исполнении (что даст большую нагрузку на решатель логики и скорее всего приведёт к превышению лимита времени его работы).
Во-вторых, наиболее вероятно, что некоторые функции в программе будут вызываться более одного раза. \autoref{algo:traditional-se} исследует функцию каждый раз, когда она вызывается. Лучшим подходом было бы собирать \emph{резюме функций} и переиспользовать их, чтобы избежать затрат на исполнение и решение ограничений. Эта техника используется в \emph{композициональном символьном исполнении}~\cite{christakis2015ic,godefroid2007compositional,lin2016compositional}. Современные движки могут резюмировать ациклические фрагменты кода и невложенные циклы~\cite{xie2016proteus}. Однако точное резюмирование произвольных функций на данный момент остаётся открытой проблемой~\cite{SurveySymExec-CSUR18}.

В данной работе представлен подход к реализации примитивов для работы с кучей (\texttt{initSymbolicState}, \texttt{readHeap}, \texttt{writeHeap}, \texttt{allocateInHeap} и \texttt{merge}). Этот подход позволяет:
\begin{enumerate*}[label=\alph*)]
    \item автоматически порождать \emph{точные условия верификации} \emph{произвольных} функций, что даёт возможность построить композициональный верификатор;
    \item порождать условия верификации не прибегая к раскрутке рекурсии, из-за чего эти условия оказываются сравнительно компактными.
\end{enumerate*}

Подход основывается на концепции \emph{композициональной символьной памяти}. В~\autoref{sec:state} представлена формальная модель композициональной символьной памяти и доказаны некоторые её свойства. В~\autoref{sec:compositional-se} приведён пример алгоритма композиционального символьного исполнения без раскрутки рекурсии. Алгоритм исследует функцию в изоляции и порождает условия верификации, которые описывают её \emph{точное} поведение и, соответственно, могут быть переиспользованны, что позволяет избежать повторного исследования функции в будущем. Затем, в~\autoref{sec:encoding} описано сведение задачи решения этих условий верификации к задаче выполнимости ограниченных дизъюнктов Хорна.
