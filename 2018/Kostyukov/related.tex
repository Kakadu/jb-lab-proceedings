\section{Смежные направления исследований}
\label{sec:related}

Символьное исполнение давно исследуется в академической среде~\cite{SurveySymExec-CSUR18}. Композициональное символьное исполнение~\cite{christakis2015ic,godefroid2007compositional,lin2016compositional} широко известная техника модульного исследования путей исполнения программы. Насколько нам известно, на данный момент не существует подходов к резюмированию рекурсивных функций и циклов с нестандартной (произвольной) структурой. Также нам неизвестны работы, содержащие исчерпывающие описание динамической памяти в контексте композиционального символьного исполнения. 

Касательно символьного исполнения без раскрутки, работа~\cite{slaby2013compact} ближе всего к нашей по духу. Этот подход резюмирует эффекты циклов при помощи шаблонов, однако он неприменим, если значения модифицированных локаций не образуют арифметическую или геометрическую прогрессию; работа с динамической памятью и рекурсивные функции не рассматриваются вовсе. Резюмирование в этой работе порождает кванторы всеобщности в SMT-запросах. Наш же подход резюмирует поведения произвольных функций, однако ценой более сложных запросов к решателям.
Идея символьных объединений заимствована из~\cite{torlak2014lightweight}.

Логика разделения~\cite{reynolds2002separation} является очень распространённым подходом к анализу программ с динамической памятью. Инструменты, реализующие композициональные подходы в логике разделения, показывают впечатляющие результаты~\cite{calcagno2011compositional,distefano2008jstar,calcagno2011infer}. Автоматический анализ формы динамической памяти (shape analysis), основанный на логике разделения, обычно ассоциирован с абстракцией. Напротив, наш подход позволяет проводить композициональный анализ \emph{данных} в динамической памяти, а не только лишь её \emph{формы}. Существует несколько подходов символьного исполнения, вдохновлённых логикой разделения (\cite{berdine2007shape,dudka2013byte,berdine2005symbolic}), однако они некомпозициональны.

Решатели дизъюнктов Хорна с ограничениями~\cite{grebenshchikov2012hsf,fedyukovich2017sampling,hoder2012generalized,hojjat2012verification,kafle2016solving,komuravelli2016smt,mcmillan2013computing,zhu2018data,albarghouthi2017probabilistic} хорошо себя показали в роли основания для верификации, будучи использованы в нескольких фреймворках верификации~\cite{gurfinkel2015seahorn,kahsai2016jayhorn,hojjat2012verification}. Существует запрос на технику кодирования императивных программ с динамической памятью в дизъюнкты Хорна с ограничениями~\cite{gurfinkel2017context,kahsai2017quantified}. Эти техники могут быть полезны для адаптации нашего подхода к реальным индустриальным языкам программирования, таким как C или Java.

%
% Current research in Horn solving is mostly concentrated on first-order case. However, the recent work~\cite{cathcart2017higher} revives the interest to higher-order constrained Horn clauses, motivated by verification of functional programs~\cite{vazou2015bounded,unno2013automating,zhu2013compositional}. We believe that our connections to precise analysis of imperative programs with heaps will be a valueable additional argument for the further development of this area.
