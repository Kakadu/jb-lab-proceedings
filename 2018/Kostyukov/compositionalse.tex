\section{Композициональное символьное исполнение без раскрутки}
\label{sec:compositional-se}

В прошлой секции была представлена КСП, которую уже можно использовать для реализации функций \texttt{initSymbolicState}, \texttt{readHeap}, \texttt{writeHeap}, \texttt{allocateInHeap} и \texttt{merge} из~\autoref{algo:traditional-se}, и получить композициональный алгоритм для нерекурсивных функций. Однако строгость нашего представления позволяет нам пойти дальше и создать композициональный подход, не прибегающий к раскрутке рекурсии.

\subsection{Обобщённые символьные кучи}
Основная идея состоит в том, чтобы ввести \emph{обобщённые символьные кучи}, или \emph{символы куч}. Один символ кучи представляет собой класс символьных куч и описывает то, \emph{как построить} отдельно взятую символьную кучу, а не её содержание.
%
\begin{align*}
	Heap ::=\,\,&\heapset \cr
	        \mid\,\,&\GMerge{\pair{guard}{Heap}^*} \cr
	        \mid\,\,&\GMutate{Heap}{\heapset} \cr
	        \mid\,\,&\GCompose{Heap}{Heap} \cr
	        \mid\,\,&\GApp{func} \cr
	        \mid\,\,&\GRec{\funcid} \cr
\end{align*}
%
Обобщённая символьная куча может быть либо обычной символьной кучей, которую мы будем называть \emph{определённой}, либо объединением обобщённых куч по непересекающимся ограничениям, либо композицией обобщённых куч, либо записью в обобщённую кучу ($\GMutate{\sigma}{\sigma'}$~--- это синхронная запись содержимого $\sigma'$ в $\sigma$; определённая куча $\sigma'$ отражает набор одиночных записей), либо эффект некоторой (возможно неизвестной) функции (которая может быть уточнена позже), либо неподвижная точка известной рекурсивной функции. Чтобы поддержать корректность, необходимо также расширить синтаксис символьных выражений:
%
\allowdisplaybreaks
\begin{align*}
	term ::=\,\,&arith \mid loc \mid func \mid (term, \ldots, term) \cr
	arith ::=\,\,&\mathbb{N} \mid arith \pm arith \mid -arith \mid LI^{arith}(\textcolor{blue}{Heap,} loc) \\
	\mid\,\,&\aunion^{arith}{(\pair{guard}{arith}^*)} \cr
	loc ::=\,\,&\text{named location} \mid (\text{concrete heap address}).\mathbb{N} \mid null \\
			   \mid\,\,&LI^{loc}(\textcolor{blue}{Heap,} loc).\mathbb{N} \mid \aunion^{loc}{(\pair{guard}{loc}^*)} \cr
	func ::=\,\,&\funcid \mid LI^{func}(\textcolor{blue}{Heap,} loc) \\
	            \mid\,\,&\aunion^{func}{(\pair{guard}{func}^*)} \cr
	guard ::=\,\,&\top \mid \bot \mid \neg guard \mid guard \land guard \mid guard \lor guard \\
				 \mid\,\,&arith = arith \mid arith < arith \mid loc = loc \cr
\end{align*}
%
Основная идея состоит в \ruquote{запоминании} источника каждого символьного значения. Такое расширение корректно: старые символьные значения $\li{x}$ теперь станут $\li{\emptyheap, x}$. Уточнение также может быть расширено: $\fillterm{\tau}{\li{\sigma, x}}=\deref{\compose{\tau}{\sigma}}{\fillterm{\tau}{x}}$. Это не нарушит свойства КСП, так как по~\autoref{thm:composition-of-read} мы имеем $\fillterm{\tau}{\deref{\sigma}{x}}=\deref{\compose{\tau}{\sigma}}{\fillterm{\tau}{x}}$.

Теперь продемонстрируем основную идею символьного исполнения без раскрутки. Пусть у нас есть рекурсивная функция
%
\begin{lstlisting}[style=demolang]
F {
	if @$g$@ then @$A$@;				// Effect of @$A$@ is @$\sigma_A$@
	else @$B$@; call F; @$C$@; // Eff. of @$B$@ and @$C$@ are @$\sigma_B$@ and @$\sigma_C$@
}
\end{lstlisting}

Очевидный способ вычислить эффект \texttt{F} заключается в том, чтобы раскручивать эту функцию до тех пор, пока инструкция \texttt{\textcolor{blue}{call} F;} не станет недостижимой (из-за накопившихся условий верификации). Этот процесс сойдётся для \ruquote{конечных} функций, и не завершится для рекурсивных функций с неограниченным числом итераций. Однако при помощи обобщённых куч мы можем автоматически построить \emph{уравнение на состояние} функции \texttt{F}.

Процесс построения начинается с исполнения \texttt{F} в изоляции. Исполнение разветвляется по ограничению $g$. Исследуя первую ветку, получаем состояние $\sigma_A$. Теперь последовательно исполняем вторую ветку: эффектом $B$ будет $\sigma_B$, далее следует рекурсивный вызов \texttt{\textcolor{blue}{call} F}. Вместо того, чтобы исследовать тело \texttt{F} заново, мы аппроксимируем сверху эффект \texttt{F}, обозначая его за $\GRec{\texttt{F}}$. Затем, состояние двух последовательных инструкций \texttt{$B$; \textcolor{blue}{call} F;} резюмируется композицией соответствующих состояний $\sigma = \gcompose{\sigma_B}{\GRec{\texttt{F}}}$. Далее движок исполняет $C$ на $\sigma$, получая резюме $\compose{\sigma}{\sigma_C}$ для всей второй ветки. Итоговым состоянием для $F$ будет объединение двух веток, то есть $\GMerge{\pair{g}{\sigma_A}, \pair{\neg g}{\gcompose{\sigma_B}{\gcompose{\grec{\texttt{F}}}{\sigma_C}}}}$. С другой стороны, эффект \texttt{F} обозначается просто $\GRec{\texttt{F}}$. Таким образом, мы получили уравнение
\begin{align*}
\GRec{\texttt{F}}=\GMerge{\pair{g}{\sigma_A}, \pair{\neg g}{\gcompose{\sigma_B}{\gcompose{\grec{\texttt{F}}}{\sigma_C}}}}
\end{align*}
%
которое точно описывает все поведения \texttt{F}. Тем самым мы свели задачу верификации программ с кучами к задаче решения систем уравнений над состояниями. В~\autoref{sec:encoding} показано, как можно решать такие системы при помощи решателей ограниченных дизъюнктов Хорна высших порядков.

Теперь можно потребовать, чтобы движок символьного исполнения также поддерживал функцию $rhs : \funcid \rightarrow Heap$, которая отображает идентификаторы рекурсивных функций в соответствующие правые части уравнений на состояния. Эта функция нам потребуется, чтобы определить \emph{отношение независимости}. Мы называем обобщённую кучу $\sigma$ и локацию $x$ независимыми, если обобщённая куча не содержит применений неизвестных функций, и чтение каждой определённой кучи внутри обобщённой даёт только ленивые инстанциации локации $x$. $\indeprec{\sigma}{x}{U} \eqdef$
\begin{align*}
% &\left\{
    &\begin{array}{rl}
    \big(\bigvee_{l\in \dom{\sigma}} \mg{x}{l}\big) \mbox{ невыполнимо}, \mbox{ если } \sigma\in\heapset \\[0.3em]
	\bigwedge_{i=1}^n \indeprec{\sigma_i}{x}{U}, \mbox{ если } \sigma=\GMerge{\pair{g_1}{\sigma_1},\ldots,\pair{g_n}{\sigma_n}} \\[0.3em]
    \indeprec{\sigma_1}{x}{U} \wedge \indeprec{\sigma_2}{x}{U}, \mbox{ если } \sigma = \GMutate{\sigma_1}{\sigma_2}\\[0.3em]
    \indeprec{\sigma_1}{x}{U} \wedge \indeprec{\sigma_2}{x}{U}, \mbox{ если } \sigma = \GCompose{\sigma_1}{\sigma_2}\\[0.3em]
    \Indeprec{rhs(f)}{x}{U \cup \{f\}}, \mbox{ если } \sigma = \GRec{f} \mbox{ и } f\not\in U \\[0.3em]
    \top, \mbox{ если } \sigma = \GRec{f} \mbox{ и } f\in U \\[0.3em]
    \bot, \mbox{ если }\sigma=\GApp{\cdot}
    \end{array}\\
    % \right.\\
&\indep{\sigma}{x} \eqdef \indeprec{\sigma}{x}{\emptyset}
\end{align*}

Из определения следует, что чтение независимой локации $x$ в обобщённой куче всегда даёт только $\li{\emptyheap, x}$. Мы не приводим строго доказательства этого факта, однако он очевидно следует из того, что выполнимые ограничения либо остаются выполнимыми, либо становятся невыполнимыми после уточнения в некоторой куче. Невыполнимые же ограничения могут только оставаться невыполнимыми.

Отношение независимости очень важно с практической точки зрения: оно даёт возможность судить о том, меняет ли рекурсивная функция некоторую локацию в памяти. Без отношения независимости движку символьного исполнения пришлось бы трактовать каждое рекурсивное состояние $\GRec{f}$ как неизвестное, порождая недетерминированные значения при каждой операции чтения. В язык программирования со статической типизацией может быть переиспользована информация о типах: например, если рекурсивная функция модифицирует только локации типа \texttt{LinkedList}, то несмотря на то, что множество всех затронутых адресов неизвестно без раскрутки, тем не менее, например, локации типа \texttt{Tree} могут быть проигнорированы.

Далее адаптируем все операции над КСП к обобщённым кучам. Чтобы избежать неопределённостей, добавим к именам старых операций индекс $^\heapset$: например, $\deref{\sigma}{x}$ теперь будет называться $\derefold{\sigma}{x}$, $\compose{\sigma}{\sigma'}$ станет $\composeold{\sigma}{\sigma'}$ и т.д.

В первую очередь расширим определение операции чтения при помощи расширения $\afind$. Напомним, что:
\begin{align*}
\derefold{\sigma}{x} = &\findold{\sigma}{x}{\emptyheap}{LI(x)} = \findold{\sigma}{x}{\emptyheap}{\derefold{\emptyheap}{x}}\\
(\composeold{\sigma}{\sigma'})(x) = &\findold{\sigma'}{x}{\sigma}{\sigma(x)} = \findold{\sigma'}{x}{\sigma}{\derefold{\sigma}{x}}
\end{align*}
%
Заметим, что в обоих случаях четвёртым аргументом $\afind^{\heapset}$ передаётся чтение третьего аргумента. Новая версия $\afind$ будет принимать только три аргумента. Четвёртый аргумент всё ещё полезен для чтения из $\agmutate$ обобщённых куч. Расширение $\afind$ на случай обобщённых куч выглядит следующим образом $\finDnew{\sigma}{x}{\tau}\eqdef$

\begin{align}\label{eq:find-new}
%\left\{
    \begin{array}{rl}
    \finDold{\sigma}{x}{\tau}{\findnew{\tau}{x}{\emptyheap}}, \mbox{ если } \sigma\in\heapset \\[0.5em]
    \union{\pair{\compose{\tau}{g_1}}{\findnew{\sigma_1}{x}{\tau}},\ldots,\pair{\compose{\tau}{g_n}}{\findnew{\sigma_n}{x}{\tau}}}, \cr
    \mbox{ если }\sigma=\GMerge{\pair{g_1}{\sigma_1},\ldots,\pair{g_n}{\sigma_n}} \\[0.5em]
    \finDold{\sigma_2}{x}{\tau}{\findnew{\sigma_1}{x}{\tau}}, \mbox{ если } \sigma = \GMutate{\sigma_1}{\sigma_2}\\[0.5em]
    \finDnew{\sigma_2}{x}{\compose{\tau}{\sigma_1}}, \mbox{ если } \sigma = \GCompose{\sigma_1}{\sigma_2}\\[0.5em]
    \findnew{\tau}{x}{\emptyheap}, \mbox{ если } \sigma = \GRec{f} \mbox{ и } \indep{\sigma}{x}\\[0.5em]
    LI(\compose{\tau}{\sigma}, x), \mbox{ иначе}
    \end{array}
%    \right.%
\end{align}

Такое расширение корректно согласно утверждениям: \autoref{thm:find-is-associative}, \autoref{thm:merge-find} и~\autoref{thm:read-of-write}. Записи обобщённых куч определим следующим образом (это корректно по~\autoref{thm:merge-write}) $\mutate{\sigma}{x}{v}\eqdef$
\begin{align*}
%\left\{
    \begin{array}{rl}
    \mutateold{\sigma}{x}{v}, \mbox{ если } \sigma\in\heapset \\[0.4em]
    \GMerge{\pair{g_1}{\mutate{\sigma_1}{x}{v}},\ldots,\pair{g_n}{\mutate{\sigma_n}{x}{v}}}, \cr
    \mbox{ если }\sigma=\GMerge{\pair{g_1}{\sigma_1},\ldots,\pair{g_n}{\sigma_n}} \\[0.4em]
    \GMutate{\sigma_1}{\mutateold{\sigma_2}{x}{v}}, \mbox{ если } \sigma = \GMutate{\sigma_1}{\sigma_2}\\[0.4em]
    \GMutate{\sigma}{\{x\mapsto v\}}, \mbox{ иначе}
    \end{array}
%    \right.%
\end{align*}
%
Чтобы обобщить композицию, необходимо обобщить уточнение. Для $\sigma\in Heap, \sigma'\in\heapset$ определим $\fillterm{\sigma}{\sigma'}$ как частичную функцию с областью определения $\fillterm{\sigma}{\dom{\sigma'}}$ и $(\fillterm{\sigma}{\sigma'})(\fillterm{\sigma}{x})= \fillterm{\sigma}{\sigma'(x)}$ для всех $x\in\dom{\sigma}$. Интуитивно, $\fillterm{\sigma}{\sigma'}$ это куча $\sigma'$ с ленивыми ячейками, заполненными из $\sigma$. Далее, композиция $\compose{\sigma}{\sigma'}\eqdef$
%
\begin{align*}
% \left\{
    \begin{array}{rl}
    \composeold{\sigma}{\sigma'}, \mbox{ если } \sigma\in\heapset,\sigma'\in\heapset \\[0.4em]
    \GMerge{\pair{\fillterm{\sigma}{g_1}}{\compose{\sigma}{\sigma_1}},\ldots,\pair{\fillterm{\sigma}{g_n}}{\compose{\sigma}{\sigma_n}}},\\[0.4em]
    \mbox{ если }\sigma'=\GMerge{\pair{g_1}{\sigma_1},\ldots,\pair{g_n}{\sigma_n}} \\[0.4em]
    \GMerge{\pair{g_1}{\compose{\sigma_1}{\sigma'}},\ldots,\pair{g_n}{\compose{\sigma_n}{\sigma'}}},\\[0.4em]
    \mbox{ если }\sigma=\GMerge{\pair{g_1}{\sigma_1},\ldots,\pair{g_n}{\sigma_n}} \\[0.4em]
    \GMutate{\compose{\sigma}{\sigma_1}}{\fillterm{\sigma}{\sigma_2}},\mbox{ если } \sigma' = \GMutate{\sigma_1}{\sigma_2}\\[0.4em]
    \GMutate{\sigma_1}{\amutate^{\heapset}(\sigma_2, \fillterm{\sigma}{\sigma'})}, \mbox{ если } \sigma = \GMutate{\sigma_1}{\sigma_2},\sigma'\in\heapset\\[0.4em]
    \GMutate{\sigma}{\fillterm{\sigma}{\sigma'}}, \mbox{ если } \sigma\neq\GMutate{\cdot}{\cdot}, \sigma'\in\heapset\\[0.4em]
    \compose{(\compose{\sigma}{\sigma_1})}{\sigma_2}, \mbox{ если } \sigma'=\GCompose{\sigma_1}{\sigma_2}\\[0.4em]
    Call(\fillterm{\sigma}{f}), \mbox{ если } \sigma'=\GApp{f}\\[0.4em]
    \GCompose{\sigma}{\sigma'}, \mbox{ иначе}
    \end{array}
    % \right.%
\end{align*}

Корректность этого определения следует из~\autoref{thm:state-composition-is-associative}, \autoref{thm:merge-right-compose}, \autoref{thm:merge-left-compose}, \autoref{thm:compose-write}. В случае $\sigma'=\GApp{f}$ мы пробуем уточнить символьную функцию $f$ из $\sigma$, и в случае успеха (если она становится конкретной), функция $Call$ (определённая ниже в~\autoref{algo:compositional-se}) применяет её эффект, иначе она возвращает символьно применение $\GApp{\fillterm{\sigma}{f}}$.

Мы опустим определение обобщённого объединения, так как оно очевидно: объединение применяет $\amerge^\heapset$ к определённым кучам и \ruquote{линеаризует} $\agmerge$ аналогично \autoref{prop:union-properties}.\ref{prop:union-nesting}

\subsection{Алгоритм}

Полученные обобщения позволяют сформулировать алгоритм композиционального символьного исполнения без раскрутки (см.~\autoref{algo:compositional-se}). Алгоритм имеет много общего с представленным ранее~\autoref{algo:traditional-se}. Отличия состоят в том, что здесь определены функции для работы с динамической памятью при помощи операций над КСП, и изменена функция $Call$.
%
\begin{algorithm}
\caption{Композициональное символьное исполнение без раскрутки}
\label{algo:compositional-se}
%\scriptsize
\begin{algorithmic}[1]
    \State \emph{ExecStatement} \ldots \Comment{  такой же, как в~\autoref{algo:traditional-se}}
	\Function{\textsc{initSymbolicState}}{} : Heap
		\State \Return{$\emptyheap$}
	\EndFunction
	
	\Function{\textsc{merge}}{cond : $guard$, $\sigma_1$ : $Heap$, $\sigma_2$ : $Heap$} : $Heap$
	    \State \Return{$\merge{\pair{\text{cond}}{\sigma_1}, \pair{\neg \text{cond}}{\sigma_2}}$}
	\EndFunction
	
	\Function{\textsc{readHeap}}{$\sigma$ : $Heap$, addr : $loc$} : Term
	    \State \Return{$\deref{\sigma}{addr}$}
	\EndFunction
	
	\Function{\textsc{writeHeap}}{$\sigma$ : $Heap$, addr : $loc$, value : $term$} : Heap
	    \State \Return{$\mutate{\sigma}{\text{addr}}{\text{value}}$}
	\EndFunction
	
	\Function{\textsc{allocateInHeap}}{$\sigma$ : $Heap$, term : $term$} : $Heap$
	    \State \assign{$addr$}{новый конкретный адрес}
	    \State \Return{$\mutate{\sigma}{\text{addr}}{\text{term}}$}
	\EndFunction
	
	\Function{Call}{f : $func$, $\sigma$ : $Heap$, $\pi$ : $PathCondition$} : $Heap$
		\If{$f$ не конкретная функция}\State \Return{$\GApp{f}$\label{algo:line:generalized-ho-app}}\EndIf
%    	
		\If{$++$counter(f) $>$ 1} \State \Return{$\GRec{f}$\label{algo:line:generalized-rec-app}}\EndIf
%    	
		\If{$f\not\in\dom{rhs}$}
			\State \assign{\emph{Errors}$_{old}$}{\emph{Errors}\label{algo:line:exploration-start}}
			\State \assign{$\sigma_f$}{\emph{ExecStatement}($f$.\emph{body}, $\emptyheap$, $\top$)}
		    \State \assign{$rhs$}{$rhs[f \mapsto \sigma_f]$}
		    \State \assign{$errs$}{$errs[f \mapsto Errors]$\label{algo:line:exploration-end}}
			\State \assign{\emph{Errors}}{\emph{Errors}$_{old}$}

		\EndIf
    	\State \assign{\emph{Errors}}{\emph{Errors} $\cup\,\,\{ \pi\wedge\fillterm{\sigma}{e} \mid e\in\textit{Errors},\textit{isSatisfiable}(\pi\wedge\fillterm{\sigma}{e}) \}$\label{algo:line:effect-errors}}
		\State \Return{$\compose{\agmerge{\pair{\pi}{\sigma}}}{rhs(f)}$\label{algo:line:effect-compose}}
	\EndFunction
	
\end{algorithmic}
\end{algorithm}


Применения высшего порядка и рекурсивные вызовы порождают соответствующие символы обобщённых куч (строки~\ref{algo:line:generalized-ho-app},~\ref{algo:line:generalized-rec-app}). Исследованные состояния запоминаются в $rhs$, а найденные ошибки в $errs$ (в настоящей реализации это могут быть базы данных условий верификации). Если функция ещё не была исследована, алгоритм исследует её в изоляции и запоминает её эффект (строки~\ref{algo:line:exploration-start}---\ref{algo:line:exploration-end}), иначе алгоритм переиспользует уже посчитанный эффект. Воспроизведение эффекта происходит за счёт вычисления обобщённой композиции (строка~\ref{algo:line:effect-compose}) и обновления информации об ошибках (строка~\ref{algo:line:effect-errors}). Композиция с $\agmerge\pair{\pi}{\sigma}$ позволяет отсечь ветки, не достижимые при текущем ограничении пути $\pi$.

\begin{exmp}\label{EXMP:HEAP-OF-PROG}
Рассмотрим применение~\autoref{algo:compositional-se} к функции \texttt{Test} из~\autoref{exmp:prog}.

Исполнение начинает с $\emptyheap$, $\dom{errs}=\dom{rhs}=\emptyset$ и разветвляется по ограничению $g = (\li{\emptyheap, node} \neq null)$. Ветка, <<защищённая>> ограничением $\neg g$, сразу обрывается с итоговым состоянием $\emptyheap$. Другая ветка исполняется с ограничением пути $g$. Содержимое кучи $\emptyheap$ в $\li{\emptyheap, node}$ считывается (заметим, что чтение не терпит неудачу, т.к. $g\wedge \neg g$ невыполнимо) и записывается по трём именованным локациям, далее первый элемент кортежа перезаписывается в $0$, что даёт
\begin{align*}
\sigma_1 = \Big\{ &visited \mapsto \Li{\emptyheap, \li{\emptyheap, node}.1};\\&oldValue \mapsto \Li{\emptyheap, \li{\emptyheap, node}.2};\\
            & next \mapsto \Li{\emptyheap, \li{\emptyheap, node}.3};\\
			& \li{\emptyheap, node}\mapsto \Big(0, \Li{\emptyheap, \li{\emptyheap, node}.2},\\ &\qquad\qquad\qquad\qquad\Li{\emptyheap, \li{\emptyheap, node}.3}\Big)\Big\}
\end{align*}

Далее, алгоритм доходит до инструкции \texttt{\textcolor{blue}{call} Inc}. Т.к. \texttt{Inc}~$\not\in\dom{rhs}$, алгоритм начинает исследование \text{Inc} в изоляции, получая
\begin{align*}
\sigma_{\texttt{Inc}} = \GRec{\texttt{Inc}} = \GMerge{&\paiR{\neg g}{\emptyheap}, \paiR{g \wedge \neg g'}{\sigma_2},\\
&\paiR{g\wedge g'}{\gcompose{\sigma_3}{\grec{\texttt{Inc}}}}},
\end{align*}
где $g' = \big(\Li{\emptyheap, \li{\emptyheap, node}.1} = 0\big)$ и
\begin{align*}
\sigma_2 = \Big\{ &visited \mapsto \Li{\emptyheap, \li{\emptyheap, node}.1};\\
&value \mapsto \Li{\emptyheap, \li{\emptyheap, node}.2};\\
&next \mapsto \Li{\emptyheap, \li{\emptyheap, node}.3};\Big\}\\
\sigma_3 = \Big\{ &visited \mapsto \Li{\emptyheap, \li{\emptyheap, node}.1};\\
&value \mapsto \Li{\emptyheap, \li{\emptyheap, node}.2};\\
&next \mapsto \Li{\emptyheap, \li{\emptyheap, node}.3};\\
			& \li{\emptyheap, node}\mapsto \Big(1, \Li{\emptyheap, \li{\emptyheap, node}.2},\\
			& \qquad\qquad\qquad\qquad\Li{\emptyheap, \li{\emptyheap, node}.3}\Big);\\
&node \mapsto \li{\emptyheap, \li{\emptyheap, node}.3}\Big\}
\end{align*}

Затем эффект \texttt{Inc} применяется на состоянии $\sigma_1$, что даёт
\begin{align*}
&\sigma_4 = \compose{\agmerge{\pair{g}{\sigma_1}}}{\sigma_{\texttt{Inc}}}=\\
&=\compose{\sigma_1}{\GMerge{\paiR{g\wedge\neg \fillterm{\sigma_1}{g}}{\emptyheap},\\ &\qquad\qquad\qquad\paiR{g\wedge\fillterm{\sigma_1}{g} \wedge \neg \fillterm{\sigma_1}{g'}}{\sigma_2},\\
&\qquad\qquad\qquad\paiR{g\wedge\fillterm{\sigma_1}{g}\wedge \fillterm{\sigma_1}{g'}}{\GCompose{\sigma_3}{\grec{\texttt{Inc}}}}}}=\\
&=\agmerge{\paiR{g}{\gcompose{(\compose{\sigma_1}{\sigma_3})}{\grec{\texttt{Inc}}}}}
\end{align*}

Последнее равенство вытекает из того, что $\fillterm{\sigma_1}{g'}=\big(\deref{\sigma_1}{\deref{\sigma_1}{node}.1} = 0\big) = \big(\deref{\sigma_1}{\li{\emptyheap, node}.1} = 0\big) = (0 = 0) = \top$ and $\fillterm{\sigma_1}{g} = g$. Далее движок читает \texttt{newValue}, получая
\begin{align*}
&\deref{\sigma_4}{\li{\emptyheap, node}.2}=\\
&\quad=\aunion{\pair{g}{\Li{\gcompose{(\compose{\sigma_1}{\sigma_3})}{\grec{\texttt{Inc}}}, \li{\emptyheap, node}.2}}}
\end{align*}

\texttt{oldValue} и $\GRec{\texttt{Inc}}$ независимы, так что движок читает \texttt{oldValue} из $\sigma_1$. Наконец, движок запоминает условие пути для \texttt{\textcolor{blue}{fail}}, которое выглядит следующим образом
\begin{align}\label{exmp:fail-pc}
&g \wedge \Li{\gcompose{(\compose{\sigma_1}{\sigma_3})}{\grec{\texttt{Inc}}}, \li{\emptyheap, node}.2}=\\\notag
&\quad=\Li{\emptyheap, \li{\emptyheap, node}.2}
\end{align}
\end{exmp}

\begin{remk*}
Несмотря на то, что мы рассмотрели КСП в контексте <<игрушечного>> языка программирования, этот формализм может быть применён к произвольному языку программирования путём расширения синтаксиса символьных выражений и переопределения операции подстановки $\fillterm{}{}$.
\begin{itemize}
\item Композиция была определена только для символьных куч. В общем случае необходимо расширить определение на полные состояния программ. Естественно поддержать стек, подстановку типовых переменных и т.д.
\item Синтаксис символьных термов содержит только примитивные значения и кортежи; множество локаций в динамической памяти не определено. Их можно расширить так, чтобы поддержать массивы, арифметику указателей, реинтерпретацию данных, анализ формы кучи и т.д. Два возможных подхода по устройству модели памяти могут быть найдены в~\cite{dudka2013byte,gurfinkel2017context}.
\item В таких языках, как Java или C\# функции могут разветвляться по операторам проверки приводимости типов во время исполнения. Например: <<приводим ли объект типа $T$ к типу $U$>>. На случай, когда типы времени исполнения неизвестны (например, когда тип локации является типовой переменной или абстрактным типов), необходимо расширить синтаксис ограничений символьной операцией $T <: U$. Исполнение программ с типовым ограничениями станет композициональным автоматически после определения правила для $\fillterm{\sigma}{(T <: U)}$, которое должно считывать информацию о типах из состояния. Если оба типа конкретизируются в контекстном состоянии, выражение можно вычислять конкретно.
\end{itemize}
\end{remk*}
