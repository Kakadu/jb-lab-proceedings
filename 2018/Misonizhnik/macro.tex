%\newsavebox\CBox
%\newcommand\hcancel[2][0.1pt]{%
%  \ifmmode\sbox\CBox{$#2$}\else\sbox\CBox{#2}\fi%
%  \makebox[0pt][l]{\usebox\CBox}%
%  \textcolor{red}{\rule[0.3\ht\CBox-#1/2]{\wd\CBox}{#1}}}

\newcommand{\vvariance}[0]{v}
\newcommand{\objecttype}[0]{\texttt{Object}}
\newcommand{\icomparable}[0]{\texttt{IComparable}}
\newcommand{\ienumerable}[0]{\texttt{IEnum}}
\newcommand{\dictionarytype}[0]{\texttt{Dictionary}}
\newcommand{\idictionarytype}[0]{\texttt{IDictionary}}
\newcommand{\iserializabletype}[0]{\texttt{ISerializable}}
\newcommand{\icollectiontype}[0]{\texttt{ICollection}}
\newcommand{\ideserializationcallbacktype}[0]{\texttt{IDeserializationCallback}}
\newcommand{\ireadonlydictionarytype}[0]{\texttt{IReadOnlyDictionary}}
\newcommand{\ireadonlycollectiontype}[0]{\texttt{IReadOnlyCollection}}
\newcommand{\tkeytype}[0]{\texttt{TKey}}
\newcommand{\tvaluetype}[0]{\texttt{TValue}}
\newcommand{\basetype}[0]{\texttt{Base}}
\newcommand{\derivedtype}[0]{\texttt{Derived}}
\newcommand{\icloneable}[0]{\texttt{ICloneable}}
\newcommand{\keyvaluepairtype}[0]{\texttt{Pair}}
\newcommand{\ttypeparam}[0]{\texttt{T}}
\newcommand{\clonemethod}[0]{\texttt{Clone}}
\newcommand{\inttype}[0]{\texttt{Int32}}
\newcommand{\valuetype}[0]{\texttt{ValueType}}
\newcommand{\iftype}[0]{\texttt{IComp}}
\newcommand{\imtype}[0]{\texttt{IBuilder}}
\newcommand{\intype}[0]{\texttt{IBox}}
\newcommand{\ptype}[0]{\texttt{Symbol}}
\newcommand{\dtype}[0]{\texttt{OneRec}}
\newcommand{\btype}[0]{\texttt{TwoRec}}
\newcommand{\ftype}[0]{\texttt{Rec}}
\newcommand{\ctype}[0]{\texttt{Thing}}
\newcommand{\xtype}[0]{{x}}
\newcommand{\ytype}[0]{{y}}
\newcommand{\ztype}[0]{{z}}
\newcommand{\stype}[0]{\texttt{Line}}
\newcommand{\ltype}[0]{\texttt{Combo}}
\newcommand{\edge}[0]{<::}
\newcommand{\subtype}[0]{<:}
\newcommand{\notsubtype}{\:\not\!\!{<:}\:}
\newcommand{\interface}[0]{interface}
\newcommand{\sealed}[0]{sealed}
\newcommand{\class}[0]{class}
\newcommand{\keywordin}[0]{in}
\newcommand{\keywordout}[0]{out}
\newcommand{\where}[0]{where}
\newcommand{\lcm}[1]{lcm(#1)}

%macro for reduction PCP

\newcommand{\pcpxtype}[0]{x}
\newcommand{\pcpytype}[0]{y}
\newcommand{\pcpztype}[0]{z}
\newcommand{\pcpotype}[0]{t}
\newcommand{\pcpptype}[0]{p}
\newcommand{\pcpqtype}[0]{q}
\newcommand{\pcprconstr}[0]{{R}}
\newcommand{\pcpsconstr}[0]{{S}}
\newcommand{\pcpeconstr}[0]{{E}}
\newcommand{\pcpfstsym}[0]{{1}}
\newcommand{\pcpsndsym}[0]{{M}}
\newcommand{\pcpuword}[1]{{U}_{#1}}
\newcommand{\pcpvword}[1]{{V}_{#1}}
\newcommand{\pcpdword}[0]{D}
\newcommand{\pcpqword}[1]{{Q}_{#1}}
\newcommand{\pcpwword}[1]{{W}_{#1}}
\newcommand{\pcpinitalt}[0]{{U}_{0}^{L}}
\newcommand{\pcpinitblt}[0]{{V}_{0}^{L}}
\newcommand{\pcpinitct}[0]{{W}_{0}}
\newcommand{\pcprestalt}[0]{{U}^{L}}
\newcommand{\pcprestblt}[0]{{V}^{L}}
\newcommand{\pcprestct}[0]{{W}}
\newcommand{\pcpinitart}[0]{{U}_{0}^{R}}
\newcommand{\pcpinitbrt}[0]{{V}_{0}^{R}}
\newcommand{\pcprestart}[0]{{U}^{R}}
\newcommand{\pcprestbrt}[0]{{V}^{R}}
\newcommand{\pcpendit}[0]{{I}}
\newcommand{\pcpendkt}[0]{{K}}
\newcommand{\pcpendlt}[0]{{L}}
\newcommand{\pcplword}[0]{{L}}
\newcommand{\pcpinitkt}[0]{{K}}
\newcommand{\pcpinitlt}[0]{{L}}
\newcommand{\pcpendmt}[0]{{M}}
\newcommand{\pcpendnt}[0]{{N}}
\newcommand{\substt}[2]{#1 \mapsto #2}
\newcommand{\subst}[2]{\substt{#1}{#2#1}}
\newcommand{\setctor}[1]{\lbrace \, #1 \, \rbrace}
\newcommand{\unionctor}[1]{\textrm{[} \, #1 \, \textrm{]}}
\newcommand{\finiteseq}[1]{#1^{<\omega}}
\newcommand{\substname}[1]{subst_{#1}}
\newcommand{\fromindextosubst}[1]{F(#1)}
\newcommand{\interpretation}[2]{I_{#1}(#2)}
\newcommand{\interpretationf}[1]{I_{#1}}


\newcommand{\typefol}{\mathcal{L}_t}
\newcommand{\typesignatfs}{\mathcal{F}_t}
\newcommand{\typesignatps}{\mathcal{P}_t}
\newcommand{\typesignatvs}{\mathcal{V}}
\newcommand{\groundtypeset}{\mathcal{GT}}
\newcommand{\purefol}{\mathcal{L}_m}
\newcommand{\puresignatfs}{\mathcal{F}_m}
\newcommand{\puresignatps}{\mathcal{P}_m}
\newcommand{\puresignatvs}{\mathcal{V}}

\newcommand{\tsealed}[2]{#1^{\scriptsize{~\setlength{\fboxsep}{1pt}\fbox{#2}}}}
\newcommand{\tinher}[2]{#1^{\scriptsize{~#2}}}

\newcommand{\csharp}{\textsc{C\#}}
\newcommand{\java}{\textsc{Java}\;}
\newcommand{\scala}{\textsc{Scala}}
\newcommand{\dotnet}{\textsc{.NET}}
\newcommand{\smtlib}{\textsc{SMT-LIB2}}
\newcommand{\zthree}{\textsc{Z3}}
\newcommand{\operis}[2]{#1\text{ }is\text{ }#2}
\newcommand{\ctor}[2]{#1\texttt{<}\overline{#2}\texttt{>}}
\newcommand{\onector}[2]{#1\texttt{<}#2\texttt{>}}
\newcommand{\substitution}[2]{[\overline{#1}/\overline{#2}]}
\newcommand{\supertype}[2]{supertype(#1, \: #2)}
\newcommand{\hcwp}{hasConstructorWithoutParameter}
\newcommand{\newvars}{newVars}
\newcommand{\newvar}{newVar}
\newcommand{\counts}{count}

\newcommand{\operatoris}{\textbf{is}}
\newcommand{\operatoras}{\textbf{as}}

\newcommand{\subtypesat}{\texttt{SUBTYPE-SAT}}
\newcommand{\pcp}{\texttt{PCP}}
\newcommand{\pcpctid}{\texttt{PCP-CT}}

\newcommand{\generalclasstable}[4]
{
    \begin{align*}
    #1 \, \edge \,\, #2_1, \ldots, #2_{#3}#4
    \end{align*}
}
\newcommand{\classtableline}[2]{
    &#1 \quad &\edge& \quad #2
}
\newcommand{\addclasstableline}[1]
{
    & \quad  && \quad #1
}
\newcommand{\tvgeneralclasstable}[4]
{
    \begin{align}\label{#4}
    #1 \, \subtype \,\, #2_1, \ldots, #2_{#3}
    \end{align}
}
\newcommand{\tvclasstableline}[2]
{
    &\qquad where \,\, #1 \, &\subtype& \,\, #2
}

\renewcommand\eqdef{\mathrel{\stackrel{\makebox[0pt]{\mbox{\normalfont\tiny def}}}{=}}}
\renewcommand\restr[2]{{% we make the whole thing an ordinary symbol
  \left.\kern-\nulldelimiterspace % automatically resize the bar with \right
  #1 % the function
  \vphantom{\big|} % pretend it's a little taller at normal size
  \right|_{#2} % this is the delimiter
  }}

\lstdefinelanguage{CSharp}
{
 morecomment = [l]{//}, 
 morecomment = [l]{///},
 morecomment = [s]{/*}{*/},
 morestring=[b]", 
 sensitive = true,
 morekeywords = {abstract,  event,  new,  struct,
   as,  explicit,  null,  switch,
   base,  extern,  object,  this,
   bool,  false,  operator,  throw,
   break,  finally,  out,  true,
   byte,  fixed,  override,  try,
   case,  float,  params,  typeof,
   catch,  for,  private,  uint,
   char,  foreach,  protected,  ulong,
   checked,  goto,  public,  unchecked,
   class,  if,  readonly,  unsafe,
   const,  implicit,  ref,  ushort,
   continue,  in,  return,  using,
   decimal,  int,  sbyte,  virtual,
   default,  interface,  sealed,  volatile,
   delegate,  internal,  short,  void,
   do,  is,  sizeof,  while, where,
   double,  lock,  stackalloc,   
   else,  long,  static,  var, 
   enum,  namespace,  string}
}

\lstdefinestyle{sharpc}{language=CSharp,
    %frame=lr,
    frame=l,
    rulecolor=\color{blue!80!black},
    basicstyle=\ttfamily\small,
    keywordstyle=\color{blue}\ttfamily,
    stringstyle=\color{red}\ttfamily,
    commentstyle=\color{green}\ttfamily,
    morecomment=[l][\color{magenta}]{\#}
}

\newcommand{\dima}[1]{$[$\textbf{\textcolor{purple}{Дима}}:~~\emph{\textcolor{darkgray}{#1}}$]$}
\newcommand{\sasha}[1]{$[$\textbf{\textcolor{blue}{Саша}}:~~\emph{\textcolor{darkgray}{#1}}$]$}
