\section{\subtypesat{} неразрешима}

\begin{Listing}%
\pcpct{}
\caption{Таблица классов \pcpctid{}}
\label{fig:pcpclasstable}
\end{Listing}

\begin{Figure}
\begin{tikzpicture}
  \matrix (m) [matrix of math nodes, row sep=2em,
    column sep=3em]{
    &&& \pcpuword{i}\#1 \\
    \pcpfstsym\#1 & \cdots & \pcpsndsym\#1  & \pcpwword{i}\#1 & \pcprconstr\#1 & \pcpsconstr\#1\\
    &&& \pcpvword{i}\#1\\};
  \path[-stealth]
    (m-1-4) edge [dashed] (m-2-4)
            edge [dashed] (m-2-5)
            edge [dashed] (m-2-6)
            edge (m-2-1)
            edge (m-2-2)
            edge (m-2-3)
    (m-3-4) edge [dashed] (m-2-4)
            edge [dashed] (m-2-5)
            edge [dashed] (m-2-6)
            edge (m-2-1)
            edge (m-2-2)
            edge (m-2-3);
\end{tikzpicture}

\caption{Граф зависимости типовых параметров для \pcpctid{}}
\label{fig:pcpvariablegraph}
\end{Figure}


Мы докажем неразрешимость задачи \subtypesat{} путём сведения к ней \emph{проблемы соответствий Поста}.
%
\begin{paragraph}{Проблема соответствий Поста}
Пусть $\lbrace (\overline{\pcpuword{1}}, \overline{\pcpvword{1}}), \ldots , (\overline{\pcpuword{n}}, \overline{\pcpvword{n}})\rbrace$ множество пар не пустых слов над конечным алфавитом $\setctor{\pcpfstsym,\ldots,\pcpsndsym}$. Проблема соответствий Поста (\pcp) состоит в том, чтобы определить, существует ли последовательность индексов $i_1, \ldots, i_r$ такая, что $\overline{\pcpuword{i_1}} \ldots \overline{\pcpuword{i_r}}= \overline{\pcpvword{i_1}} \ldots \overline{\pcpvword{i_r}}$.
\end{paragraph}
%


Известно, что \pcp{} неразрешима~\cite{post1946variant}.

Для сведения зафиксируем таблицу классов~\autoref{fig:pcpclasstable}. Заметим, что она нерасширяющаяся, так как граф зависимости типовых параметров~\autoref{fig:pcpvariablegraph} не имеет циклов. Также таблица классов не содержит контравариантных конструкторов.

Рассмотрим задачу \subtypesat{} для формулы $\psi$ с типовыми параметрами $\pcpxtype, \, \pcpytype, \, \pcpztype, \, \pcpqtype, \, \pcpptype, \, \pcpotype$ и таблицей классов \pcpctid{}, где $\psi$ определяется следующим образом:
% \[
%     \pcpinitat\pcpxtype \subtype \pcpxtype \land \pcpendkt\pcpxtype \subtype \pcpxtype \land \pcpinitbt\pcpytype \subtype \pcpytype \land \pcpendlt\pcpytype \subtype \pcpytype \land \pcpinitct\pcpztype \subtype \pcpztype \land \pcpxtype \subtype \pcpztype \land \pcpytype \subtype \pcpztype
% \]
\[
    \begin{array}{l}
    \phi_{0} \eqdef \pcprconstr_{0} \subtype \pcpptype \land \pcpsconstr_{0} \subtype \pcpqtype \land \pcprestct \subtype \pcpztype \\
    \phi_{1} \eqdef \pcprestalt \subtype \pcpxtype \land \pcprestart \subtype \pcpxtype \land \pcpxtype \subtype \pcpptype \land \pcpxtype \subtype \pcpqtype \land \pcpxtype \subtype \pcpztype \land \phi_{0} \\
    \phi_{2} \eqdef \pcprestblt \subtype \pcpytype \land \pcprestbrt \subtype \pcpytype \land \pcpytype \subtype \pcpptype \land \pcpytype \subtype \pcpqtype \land \pcpytype \subtype \pcpztype \land \phi_{0} \\
    \psi \eqdef \pcpdword \subtype \pcpotype \land \pcpxtype \subtype \pcpotype \land \pcpytype \subtype \pcpotype \land \pcpotype \notsubtype \pcpeconstr \land \phi_{1} \land \phi_{2}
    \end{array}
\]

Основная идея сведения в том, чтобы представить слова из $\setctor{\pcpfstsym{},\ldots,\pcpsndsym{}}^{+}$ в виде цепочки ковариантных конструкторов, оканчивающихся $\pcpeconstr$. Например, слова $\overline{\pcpuword{i}}$ и $\overline{\pcpvword{j}}$ представляются как $\overline{\pcpuword{i}}\pcpeconstr$ и $\overline{\pcpvword{j}}\pcpeconstr$. Перечисление потенциальных решений \pcp{} кодируется в \pcpctid{} и $\psi$. Мы покажем недетерминированный процесс, при котором, заменяя типовые переменные на типовые конструкторы, применённые к новым переменным, и упрощая полученную формулу, мы будем последовательно уточнять типовые переменные в формуле $\psi$.

\begin{defn}
    \emph{Закрытая} подстановка --- это подстановка, у которой в образе лежат только закрытые типы. Подстановка является \emph{полной} для формулы $f$, если её домен в точности является множеством типовых переменных из формулы $f$.
\end{defn}
К подстановкам можно применить операцию \emph{композиции}. Например,
\[
    \substt{\pcpxtype}{C\pcpxtype'} \circ \substt{\pcpxtype'}{D\pcpxtype''} = \substt{\pcpxtype}{C}{D\pcpxtype''}
\]

Также определим композицию для \emph{множеств подстановок}:
\[
    \setctor{\substname{1}^{left}; \, \ldots; \,  \substname{n}^{left}} \circ \setctor{\substname{1}^{right}; \, \ldots; \,  \substname{k}^{right}} \eqdef
\]
\[
    \bigcup_{\mathclap{1\leq i \leq n, \, 1
    \leq j \leq k}} \, \setctor{\substname{i}^{left} \circ \substname{k}^{right}}
\]
% \end{defn}
Для упрощения нотации мы будем опускать штрихи при записи новых переменных, позволяя использовать одинаковые имена переменных и в домене, и в образе. Например, вместо $\substt{\pcpxtype}{C\pcpxtype'}$ мы будем писать $\subst{\pcpxtype}{C}$.

Заметим, что $\psi$ является конъюнкцией атомов, а значит выполнимость $\psi$ влечёт выполнимость каждой из подформул. Теперь рассмотрим полные подстановки, которые не обращают формулу $\phi_{0}$ в ложь.

\begin{lem}{\label{lem:pcpconstr}}
Для формулы
\[
    \phi_{0} \eqdef \pcprconstr_{0} \subtype \pcpptype \land \pcpsconstr_{0} \subtype \pcpqtype \land \pcprestct \subtype \pcpztype,
\]
только композиция подстановок
\[
    \begin{array}{l}
        \subst{\pcpptype}{\pcprconstr}, \, \substt{\pcpptype}{\pcpeconstr}, \, \substt{\pcpptype}{\pcprconstr_{0}}, \, \\
        [0.6em] \\
        \subst{\pcpqtype}{\pcpsconstr}, \, \substt{\pcpqtype}{\pcpeconstr}, \, \substt{\pcpqtype}{\pcpsconstr_{0}}, \, \\
        [0.6em] \\
        \substt{\pcpztype}{\pcpwword{i}, \, \substt{\pcpztype}{\pcpeconstr}, \, \subst{\pcpztype}{\pcprestct}}
    \end{array}
\] не обращает её в ложь.
\end{lem}
\begin{proof*}
    Типовой конструктор $\pcprconstr_{0}$ имеет только один номинальный надтип с головными конструкторами $\pcprconstr_{0}, \, \pcprconstr$ и $\pcpeconstr$. Следовательно, после применения подстановки отличной от $\setctor{\subst{\pcpptype}{\pcprconstr}; \, \substt{\pcpptype}{\pcpeconstr}; \, \substt{\pcpptype}{\pcprconstr_{0}}}$ к формуле $\phi_1$, правила \textsc{Var} и \textsc{Super} не могут быть применены к формуле, чтобы упростить её, поэтому $\phi_1$ обращается в ложь. Применение подстановок $\substt{\pcpptype}{\pcpeconstr}$ и $\substt{\pcpptype}{\pcprconstr_{0}}$ превращают в истину атом $\pcprconstr_{0} \subtype \pcpptype$. Применение $\subst{\pcpptype}{\pcprconstr}$ превращает атом в самого себя. Следовательно, только композиция этих подстановок не обращает формулу в ложь.
    
    Аналогичные рассуждения можно провести для типовых переменных $\pcpqtype$ и $\pcpztype$.
\qed\end{proof*}
%
\begin{lem}{\label{lem:pcpphi1}}
Для формулы
\[
    \phi_{1} \eqdef \pcprestalt \subtype \pcpxtype \land \pcprestart \subtype \pcpxtype \land \pcpxtype \subtype \pcpptype \land \pcpxtype \subtype \pcpqtype \land \pcpxtype \subtype \pcpztype \land \phi_{0},
\]
только композиция полных подстановок
\[
    \unionctor{\subst{\pcpptype}{\pcprconstr}; \, \subst{\pcpqtype}{\pcpsconstr}; \,
    \subst{\pcpztype}{\pcpwword{i}}; \,
    \subst{\pcpxtype}{\pcpuword{i}}}
\]
\[
    \unionctor{\substt{\pcpptype}{\pcpeconstr}; \, \substt{\pcpqtype}{\pcpeconstr}; \,
    \substt{\pcpztype}{\pcpeconstr}; \, \substt{\pcpxtype}{\pcpeconstr}}
\] не обращает её в ложь.
\end{lem}
\begin{proof*}
Для доказательства рассмотрим по отдельности все возможные подстановки.

Так как $\phi_{0}$ является подформулой $\phi_1$, \autoref{lem:pcpconstr}~влечёт, что 
кандидаты на подстановку в $\pcpztype$, которые не будут сразу же обращать формулу $\phi_{1}$ в ложь, содержатся в множестве \[\setctor{\subst{\pcpztype}{\pcpwword{i}}; \, \substt{\pcpztype}{\pcpeconstr}; \, \substt{\pcpztype}{\pcprestct}}\].

\begin{itemize}
    \item 
        Рассмотрим подстановку $\substt{\pcpztype}{\pcprestct}$. Её применение к формуле $\phi_{1}$ порождает, в частности, следующую конъюнкцию:
        \[
            \pcprestalt \subtype \pcpxtype \land \pcpxtype \subtype \pcprestct
        \]
        По транзитивности подтипирования эта конъюнкция влечёт выполнимость атома $\pcprestalt \subtype \pcprestct$, который упрощается в ложь. Следовательно, подстановка содержащая $\substt{\pcpztype}{\pcprestct}$ обращает формулу в $\phi_1$.
    \item
        Рассмотрим подстановку $\subst{\pcpztype}{\pcpwword{i}}$. Её применение к $\phi_{1}$ порождает, в частности, следующие конъюнкции:
        \[
         \pcprestalt \subtype \pcpxtype \land \pcprestart \subtype \pcpxtype \land \pcpxtype \subtype \pcpwword{i}\pcpztype
        \]
        Для того, чтобы эта формулы была выполнима, подстановка должна отобразить $\pcpxtype$ в общий надтип для $\pcprestalt$ и $\pcprestart$, и у него быть номинальный надтип сконструированный головным конструктором $\pcpwword{i}$. Единственными такими подстановками будут $\setctor{\subst{\pcpxtype}{\pcpuword{i}}; \, \subst{\pcpxtype}{\pcpwword{i}}}$.
        \begin{itemize}
            \item 
                Рассмотрим подстановку $\subst{\pcpxtype}{\pcpwword{i}}$. Её применение к $\phi_{1}$ порождает. в частности, следующую конъюнкцию:
                \[
                 \pcprconstr_{0} \subtype \pcpptype \land \pcpwword{i}\pcpxtype \subtype \pcpptype
                \]
                Эта формула содержит атом $\pcpwword{i}\pcpxtype \subtype \pcpptype$, который не обращается в ложь только при применении подстановки $\subst{\pcpptype}{\pcpwword{i}}$ к типовой переменной $\pcpptype$. Но, по~\autoref{lem:pcpconstr}, кандидатами на подстановку в $\pcpptype$ в $\pcpptype$ являются \[\setctor{\subst{\pcpptype}{\pcprconstr}; \, \substt{\pcpptype}{\pcpeconstr}; \, \substt{\pcpptype}{\pcprconstr_{0}}}\]. Следовательно, любая подстановка, содержащая $\unionctor{\subst{\pcpxtype}{\pcpwword{i}}; \, \subst{\pcpztype}{\pcpwword{i}}}$, обращает $\phi_1$ в ложь.
            \item
                Рассмотрим подстановку  $\subst{\pcpxtype}{\pcpuword{i}}$. Её применение к  $\phi_{1}$ порождает следующую конъюнкцию:
                \[
                 \pcprconstr_{0} \subtype \pcpptype \land \pcpuword{i}\pcpxtype \subtype \pcpptype \land \pcpsconstr_{0} \subtype \pcpqtype \land \pcpuword{i}\pcpxtype \subtype \pcpqtype \land \pcpuword{i}\pcpxtype \subtype \pcpwword{i}\pcpztype
                \]
                Общий надтип для $\pcprconstr_{0}$ и $\pcpuword{i}$ может иметь только один головной конструктор $\pcprconstr$. Аналогично, общий надтип для  $\pcpsconstr_{0}$ и $\pcpuword{i}$ может быть сконструирован только конструктором $\pcpsconstr$. Следовательно, подстановки $\subst{\pcpptype}{\pcprconstr}$ и $\subst{\pcpqtype}{\pcpsconstr}$ не обращают формулу $\phi_{1}$ в ложь, а подстановки из множества
                \[
                    \begin{array}{l}
                        \unionctor{\subst{\pcpztype}{\pcpwword{i}}; \, \subst{\pcpxtype}{\pcpuword{i}}} \circ
                        \setctor{\substt{\pcpptype}{\pcpeconstr}; \, \substt{\pcpptype}{\pcprconstr_{0}}} \circ \setctor{\substt{\pcpqtype}{\pcpeconstr}; \, \substt{\pcpqtype}{\pcpsconstr_{0}}}
                    \end{array}
                \] обращают $\phi_1$ в ложь.
                Из этого следует, что только подстановки
                \[
                    \unionctor{\subst{\pcpztype}{\pcpwword{i}}; \, \subst{\pcpxtype}{\pcpuword{i}}; \, \subst{\pcpptype}{\pcprconstr}; \, \subst{\pcpqtype}{\pcpsconstr}}
                \] не обращают формулу $\phi_{1}$ в ложь.
        \end{itemize}
    \item
        Рассмотрим подстановку $\substt{\pcpztype}{\pcpeconstr}$. Её применение к формуле $\phi_{1}$ порождает следующую конъюнкцию:
        \[
         \pcprestalt \subtype \pcpxtype \land \pcprestart \subtype \pcpxtype \land \pcpxtype \subtype \pcpeconstr
        \]
        Для того чтобы формула была выполнима, подстановка должна отобразить $\pcpxtype$ в общий надтип для $\pcprestalt$ и $\pcprestart$, у которого будет надтип $\pcpeconstr$. Единственной подходящий подстановкой будет $\substt{\pcpxtype}{\pcpeconstr}$. Применение подстановки \[\unionctor{\substt{\pcpztype}{\pcpeconstr}; \, \substt{\pcpxtype}{\pcpeconstr}}\] к $\phi_{1}$ даст формулу
        \[
            \begin{array}{l}
                \pcprestalt \subtype \pcpeconstr \land \pcprestart \subtype \pcpeconstr \land \pcpeconstr \subtype \pcpptype \land \pcpeconstr \subtype \pcpqtype \land \\[1em]\\ \pcpeconstr \subtype \pcpeconstr \land \pcprconstr_{0} \subtype \pcpptype \land \pcpsconstr_{0} \subtype \pcpqtype \land \pcprestct \subtype \pcpeconstr,
            \end{array}
        \] которая упрощается в 
        \[
            \pcpeconstr \subtype \pcpptype \land \pcpeconstr \subtype \pcpqtype \land \pcprconstr_{0} \subtype \pcpptype \land \pcpsconstr_{0} \subtype \pcpqtype.
        \]
        Для того, чтобы эта формула была выполнима, подстановка должна отобразить $\pcpptype$ в общий надтип для $\pcprconstr_{0}$ и $\pcpeconstr$, а $\pcpqtype$ в общий надтип для $\pcpsconstr_{0}$ и $\pcpeconstr$. Единственной подходящей подстановкой будет $\unionctor{\substt{\pcpptype}{\pcpeconstr}; \, \substt{\pcpqtype}{\pcpeconstr}}$. Следовательно, подстановка
        \[
            \unionctor{\substt{\pcpztype}{\pcpeconstr}; \, \substt{\pcpxtype}{\pcpeconstr}; \, \substt{\pcpptype}{\pcpeconstr}; \, \substt{\pcpqtype}{\pcpeconstr}}
        \] не обращает формулу $\phi_{1}$ в ложь.
        
\end{itemize}
Мы рассмотрели все возможные варианты, среди которых только подстановки
\[
    \unionctor{\subst{\pcpztype}{\pcpwword{i}}; \, \subst{\pcpxtype}{\pcpuword{i}}; \, \subst{\pcpptype}{\pcprconstr}; \, \subst{\pcpqtype}{\pcpsconstr}}
\]
\[
    \unionctor{\substt{\pcpztype}{\pcpeconstr}; \, \substt{\pcpxtype}{\pcpeconstr}; \, \substt{\pcpptype}{\pcpeconstr}; \, \substt{\pcpqtype}{\pcpeconstr}}
\] не обращают формулу в ложь.
\qed\end{proof*}
%
\begin{lem}{\label{lem:pcpphi2}}
Для формулы
\[
    \phi_{2} \eqdef \pcprestblt \subtype \pcpytype \land \pcprestbrt \subtype \pcpytype \land \pcpytype \subtype \pcpptype \land \pcpytype \subtype \pcpqtype \land \pcpytype \subtype \pcpztype \land \phi_{0},
\]
Только композиции полных подстановок
\[
    \unionctor{\subst{\pcpptype}{\pcprconstr}; \, \subst{\pcpqtype}{\pcpsconstr}; \,
    \subst{\pcpztype}{\pcpwword{i}}; \,
    \subst{\pcpytype}{\pcpvword{i}}}
\]
\[
    \unionctor{\substt{\pcpptype}{\pcpeconstr}; \, \substt{\pcpqtype}{\pcpeconstr}; \,
    \substt{\pcpztype}{\pcpeconstr}; \, \substt{\pcpytype}{\pcpeconstr}}
\] не обращают формулу в ложь.
\end{lem}
\begin{proof*}
Аналогично доказательству~\autoref{lem:pcpphi1}.
\qed\end{proof*}
\autoref{lem:pcpphi1}~и~\autoref{lem:pcpphi2} влекут, что только композиции подстановок
\[
    \unionctor{\subst{\pcpptype}{\pcprconstr}; \, \subst{\pcpqtype}{\pcpsconstr}; \,
    \subst{\pcpztype}{\pcpwword{i}}; \,
    \subst{\pcpxtype}{\pcpuword{i}}; \,
    \subst{\pcpytype}{\pcpvword{i}}}
\]
\[
    \unionctor{\substt{\pcpptype}{\pcpeconstr}; \, \substt{\pcpqtype}{\pcpeconstr}; \,
    \substt{\pcpztype}{\pcpeconstr}; \, \substt{\pcpxtype}{\pcpeconstr}; \,
    \substt{\pcpytype}{\pcpeconstr}}
\]
не обращают формулу $\phi_{1} \land \phi_{2}$ в ложь.

$\phi_{1} \land \phi_{2}$ имеет очень важное свойство: применение к ней подстановки 
\[
    \unionctor{\subst{\pcpptype}{\pcprconstr}; \, \subst{\pcpqtype}{\pcpsconstr}; \,
    \subst{\pcpztype}{\pcpwword{i}}; \,
    \subst{\pcpxtype}{\pcpuword{i}}; \,
    \subst{\pcpytype}{\pcpvword{i}}}
\]
и упрощение полученной формулы обращают её в саму себя:
\[
    \begin{array}{l}
    \pcprestalt \subtype \pcpuword{i}\pcpxtype \land
    \pcprestart \subtype \pcpuword{i}\pcpxtype \land
    \pcpuword{i}\pcpxtype \subtype \pcprconstr\pcpptype \land \\
    \pcpuword{i}\pcpxtype \subtype \pcpsconstr\pcpqtype \land \pcpuword{i}\pcpxtype \subtype \pcpwword{i}\pcpztype \land \\
    \pcprestblt \subtype \pcpvword{i}\pcpytype \land
    \pcprestbrt \subtype \pcpvword{i}\pcpytype \land
    \pcpvword{i}\pcpytype \subtype \pcprconstr\pcpptype \land \\
    \pcpvword{i}\pcpytype \subtype \pcpsconstr\pcpqtype \land \pcpvword{i}\pcpytype \subtype \pcpwword{i}\pcpztype \land \\
    \pcprconstr_{0} \subtype \pcprconstr\pcpptype \land \pcpsconstr_{0} \subtype \pcpsconstr\pcpqtype \land \pcprestct \subtype \pcpwword{i}\pcpztype \\
    \Leftrightarrow \\
    \pcpuword{i}\pcprestalt \subtype \pcpuword{i}\pcpxtype \land \pcpuword{i}\pcprestart \subtype \pcpuword{i}\pcpxtype \land \pcprconstr\pcpxtype \subtype \pcprconstr\pcpptype \land \\
    \pcpsconstr\pcpxtype \subtype \pcpsconstr\pcpqtype \land \pcpwword{i}\pcpxtype \subtype \pcpwword{i}\pcpztype \land \\
    \pcpvword{i}\pcprestblt \subtype \pcpvword{i}\pcpytype \land \pcpvword{i}\pcprestbrt \subtype \pcpvword{i}\pcpytype \land \pcprconstr\pcpytype \subtype \pcprconstr\pcpptype \land \\
    \pcpsconstr\pcpytype \subtype \pcpsconstr\pcpqtype \land \pcpwword{i}\pcpytype \subtype \pcpwword{i}\pcpztype \land \\
    \pcprconstr\pcprconstr_{0} \subtype \pcprconstr\pcpptype \land \pcpsconstr\pcpsconstr_{0} \subtype \pcpsconstr\pcpqtype \land \pcpwword{i}\pcprestct \subtype \pcpwword{i}\pcpztype \\
    \Leftrightarrow \\
    \pcprestalt \subtype \pcpxtype \land \pcprestart \subtype \pcpxtype \land \pcpxtype \subtype \pcpptype \land \pcpxtype \subtype \pcpqtype \land \pcpxtype \subtype \pcpztype \land \\
    \pcprestblt \subtype \pcpytype \land \pcprestbrt \subtype \pcpytype \land \pcpytype \subtype \pcpptype \land \pcpytype \subtype \pcpqtype \land \pcpytype \subtype \pcpztype \land \\
    \pcprconstr_{0} \subtype \pcpptype \land \pcpsconstr_{0} \subtype \pcpqtype \land \pcprestct \subtype \pcpztype \\
    = \phi_{1} \land \phi_{2}
    \end{array}
\]
Также заметим, что применение подстановки
\[
    \unionctor{\substt{\pcpptype}{\pcpeconstr}; \, \substt{\pcpqtype}{\pcpeconstr}; \,
    \substt{\pcpztype}{\pcpeconstr}; \, \substt{\pcpxtype}{\pcpeconstr}; \,
    \substt{\pcpytype}{\pcpeconstr}}
\]
и упрощение полученной формулы обращает $\phi_{1} \land \phi_{2}$ в истину.

Пусть  $N=\setctor{1,\ldots,n}$. $\finiteseq{N}$ будет обозначать множество конечных последовательностей из $N$. Для $J \in \finiteseq{N}, \, J = j_1 \ldots j_r$, будем обозначать $\pcpuword{J}T$ цепочку типовых конструкторов $\pcpuword{j_{1}} \ldots \pcpuword{j_{r}}T$. Аналогично определим $\pcpvword{J}T$ и $\pcpwword{J}T$. Иногда мы будем писать $J^r$, чтобы подчеркнуть, что $r$ является длиной $J$.
\begin{thm}{\label{thm:pcpindices}}
Формула $\phi_{1} \land \phi_{2}$ имеет множество моделей с интерпретациями $I_{J}$ такими, что
\[
    \begin{array}{c}
        I_{J}(\pcpxtype) = \pcpuword{J}\pcpeconstr, \quad I_{J}(\pcpytype) = \pcpvword{J}\pcpeconstr, \quad I_{J}(\pcpztype) = \pcpwword{J}\pcpeconstr, \\
        [0.6em] \\
        I_{J}(\pcpptype) = \pcprconstr^{r}\pcpeconstr, \quad I_{J}(\pcpqtype) = \pcpsconstr^{r}\pcpeconstr
        
    \end{array}
\]
где $J \in \finiteseq{N}$, и $r$ является длиной $J$.
\end{thm}
\begin{proof*}
Как было показано, только композиции следующих подстановок не обращают формулу $\phi_{1} \land \phi_{2}$ в ложь:
\[
\begin{array}{l}

    \substname{i} \eqdef\unionctor{\subst{\pcpptype}{\pcprconstr}; \, \subst{\pcpqtype}{\pcpsconstr}; \,
    \subst{\pcpztype}{\pcpwword{i}}; \,
    \subst{\pcpxtype}{\pcpuword{i}}; \,
    \subst{\pcpytype}{\pcpvword{i}}} \\
    [0.6em] \\
    \substname{end} \eqdef \unionctor{\substt{\pcpptype}{\pcpeconstr}; \, \substt{\pcpqtype}{\pcpeconstr}; \,
    \substt{\pcpztype}{\pcpeconstr}; \, \substt{\pcpxtype}{\pcpeconstr}; \,
    \substt{\pcpytype}{\pcpeconstr}}

\end{array}
\]
Интерпретацию можно рассматривать как закрытую подстановку, применение которой обращает формулу в истину. Так как $\substname{i}$ не изменяет $\phi_{1} \land \phi_{2}$ после применения, а $\substname{end}$ обращает её в истину, подстановка, которая обращает формулу в истину, может быть только композицией конечного числа подстановок вида  $\substname{i}$ оканчивающихся закрытой подстановкой $\substname{end}$, то есть $\substname{j_1} \circ \ldots \circ \substname{j_r} \circ \substname{end}$.

Таким образом, только замкнутые подстановки
\[
    \begin{array}{l}
    I_{J} =
    \substname{j_1} \circ \ldots \circ \substname{j_r} \circ \substname{end} = \\
    [0.6em] \\
    = \unionctor{
    \substt{\pcpptype}{\pcprconstr^{r}\pcpeconstr}; \, \substt{\pcpqtype}{\pcpsconstr^{r}\pcpeconstr}; \,
    \substt{\pcpztype}{\pcpwword{J}\pcpeconstr}; \, \substt{\pcpxtype}{\pcpuword{J}\pcpeconstr}; \,
    \substt{\pcpytype}{\pcpvword{J}\pcpeconstr}}
    \end{array}
\]
обращают формулу $\phi_{1} \land \phi_{2}$ в истину, а значит только они являются её моделями.
\qed\end{proof*}

\begin{lem}{\label{lem:pcpuletter}}
    Пусть $\overline{\pcplword{}}$ является цепочкой <<буквенных>> конструкторов, то есть конструкторов из  $\{\pcpfstsym{},\ldots,\pcpsndsym{}\}$, $J^r\in \finiteseq{N}$ с  $r > 0$. Тогда
\[
    \pcpuword{J}\pcpeconstr \subtype \overline{\pcplword{}}\pcpeconstr \lor \pcpuword{J}\pcpeconstr \subtype \overline{\pcplword{}}\pcpdword
\]
выполнима тогда и только тогда, когда
\[
     \overline{\pcpuword{j_1}}\ldots\overline{\pcpuword{j_r}}\pcpeconstr = \overline{\pcplword{}}\pcpeconstr
\]
\end{lem}
\begin{proof*}
Докажем требуемое индукцией по $r$.

\begin{paragraph}{База: $r = 1$}
\[
    \pcpuword{j_1}\pcpeconstr \subtype \overline{\pcplword{}}\pcpeconstr \lor \pcpuword{j_1}\pcpeconstr \subtype \overline{\pcplword{}}\pcpdword \Leftrightarrow \overline{\pcpuword{j_1}}\pcpeconstr \subtype \overline{\pcplword{}}\pcpeconstr \lor \overline{\pcpuword{j_1}}\pcpeconstr \subtype \overline{\pcplword{}}\pcpdword
\]
Так как $\overline{\pcpuword{j_1}}\pcpeconstr$, $\overline{\pcplword{}}\pcpeconstr$ и $\overline{\pcplword{}}\pcpdword$ сконструированы ковариантными типовыми конструкторами $\pcpfstsym,\ldots,\pcpsndsym$ без правой части в таблице классов (то есть без строгих номинальных надтипов), мы можем заключить, что
\[
    \begin{array}{l}
        \overline{\pcpuword{j_1}}\pcpeconstr \subtype \overline{\pcplword{}}\pcpeconstr \lor \overline{\pcpuword{j_1}}\pcpeconstr \subtype \overline{\pcplword{}}\pcpdword
        \Leftrightarrow \\
        [0.6em] \\
        \overline{\pcpuword{j_1}}\pcpeconstr = \overline{\pcplword{}}\pcpeconstr \lor \overline{\pcpuword{j_1}}\pcpeconstr = \overline{\pcplword{}}\pcpdword
        \Leftrightarrow \overline{\pcpuword{j_1}}\pcpeconstr = \overline{\pcplword{}}\pcpeconstr
    \end{array}
\]
\end{paragraph}

\begin{paragraph}{Переход:}
Пусть $r = k + 1$, $J = j_1 \cdot J'$, а у $J'$ длина $k$.
\begin{align*}
    &\pcpuword{j_1}\pcpuword{J'}\pcpeconstr \subtype \overline{\pcplword{}} \pcpeconstr \lor \pcpuword{j_1}\pcpuword{J'}\pcpeconstr \subtype \overline{\pcplword{}}\pcpdword \Leftrightarrow \\
     \Leftrightarrow &\overline{\pcpuword{j_1}}\pcpuword{J'}\pcpeconstr \subtype \overline{\pcplword{}} \pcpeconstr \lor \overline{\pcpuword{j_1}}\pcpuword{J'}\pcpeconstr \subtype \overline{\pcplword{}}\pcpdword \Leftrightarrow
\end{align*}
%

Так как  $\overline{\pcpuword{j_1}}\pcpeconstr$, $\overline{\pcplword{}}\pcpeconstr$ и $\overline{\pcplword{}}\pcpdword$ сконструированы ковариантными типовыми конструкторами $\pcpfstsym,\ldots,\pcpsndsym$, которые не имеют строгих номинальных надтипов, мы можем требовать $\overline{\pcplword{}}=\overline{\pcpuword{j_1}}\overline{\pcplword{}'}$.
%\[
\begin{align*}
     \Leftrightarrow&\overline{\pcpuword{j_1}}\pcpuword{J'}\pcpeconstr \subtype \overline{\pcpuword{j_1}}\overline{\pcplword{}'}\pcpeconstr \lor \overline{\pcpuword{j_1}}\pcpuword{J'}\pcpeconstr \subtype \overline{\pcpuword{j_1}}\overline{\pcplword{}'}\pcpdword \Leftrightarrow \\
     \Leftrightarrow&\pcpuword{J'}\pcpeconstr \subtype \overline{\pcplword{}'}\pcpeconstr \lor \pcpuword{J'}\pcpeconstr \subtype \overline{\pcplword{}'}\pcpdword \Leftrightarrow \textit{\small{(I.H.)}}\\
     \Leftrightarrow&\overline{\pcpuword{j_2}}\ldots\overline{\pcpuword{j_r}}\pcpeconstr = \overline{\pcplword{}'}\pcpeconstr \Leftrightarrow \overline{\pcpuword{j_1}}\overline{\pcpuword{j_2}}\ldots\overline{\pcpuword{j_r}}\pcpeconstr = \overline{\pcplword{}}\pcpeconstr
\end{align*}
%\]
\end{paragraph}
%
\qed\end{proof*}

\begin{lem}{\label{lem:pcpvletter}}
Пусть $\overline{\pcplword{}}$ является цепочкой конструкторов из $\{\pcpfstsym{},\ldots,\pcpsndsym{}\}$, $J^r\in \finiteseq{N}$ с $r > 0$. Тогда
\[
    \pcpvword{J}\pcpeconstr \subtype \overline{\pcplword{}}\pcpeconstr \lor \pcpvword{J}\pcpeconstr \subtype \overline{\pcplword{}}\pcpdword
\]
выполнима тогда и только тогда, когда
\[
     \overline{\pcpvword{j_1}}\ldots\overline{\pcpvword{j_r}}\pcpeconstr = \overline{\pcplword{}}\pcpeconstr
\]
\end{lem}
\begin{proof*}
Аналогично доказательству~\autoref{lem:pcpuletter}.
\qed\end{proof*}

\begin{lem}{\label{lem:pcpphij}}
    Формула
\[
    \phi_{J^r} \eqdef \pcpdword \subtype \pcpotype \land \pcpuword{J}\pcpeconstr \subtype \pcpotype \land \pcpvword{J}\pcpeconstr \subtype \pcpotype \land \pcpotype \notsubtype \pcpeconstr
\]
выполнима тогда и только тогда, когда $r > 0$ и
\[
 \phi'_{J} \eqdef \overline{\pcpvword{j_1}}\ldots\overline{\pcpvword{j_r}}\pcpeconstr = \overline{\pcpuword{j_1}}\ldots\overline{\pcpuword{j_r}}\pcpeconstr
\] является истиной.
\end{lem}
\begin{proof*}
Пусть $J$ является пустой последовательностью. Тогда $\phi_{J}$ становится
\[
    \pcpdword \subtype \pcpotype \land \pcpeconstr \subtype \pcpotype \land \pcpotype \notsubtype \pcpeconstr.
\]
$\pcpdword$ и $\pcpeconstr$ имеют только один общий надтип $\pcpeconstr$. Но подстановка $\pcpeconstr$ в $\pcpotype$ обратит формулу в ложь из-за атома $\pcpotype \notsubtype \pcpeconstr$. Следовательно, если $J$ пустая последовательность, то $\phi_{J}$ невыполнима.

Пусть $J$ является непустой последовательностью.

$(\Rightarrow)$

Пусть $\phi_{J}$ выполнима. Тогда $\pcpdword$ и $\pcpuword{J}\pcpeconstr$ должны иметь общий надтип. Это не может быть $\pcpeconstr$, так как для всех $i$, $\pcpeconstr$ не является надтипом для $\pcpuword{i}$. Рассмотрим все остальные надтипы $\pcpdword$. Они являются цепочками конструкторов из $\setctor{\pcpfstsym,\ldots,\pcpsndsym}$, оканчивающимися либо $\pcpdword$, либо $\pcpeconstr$. Другими словами, единственными кандидатами на роль надтипа будут $\overline{\pcplword{}}\pcpdword$ и $\overline{\pcplword{}}\pcpeconstr$, где $\overline{\pcplword{}}$ непустая цепочка конструкторов из $\setctor{\pcpfstsym,\ldots,\pcpsndsym}$.

По~\autoref{lem:pcpuletter}~и~\autoref{lem:pcpvletter}, если $I_{J}$ является интерпретацией для формулы $\phi_{J}$, то
\[
I_{J}(\pcpotype) = \overline{\pcpuword{j_1}} \ldots \overline{\pcpuword{j_r}}\pcpeconstr = \overline{\pcpvword{j_1}} \ldots \overline{\pcpvword{j_r}}\pcpeconstr
\]
Так как $\phi_{J}$ выполнима, значит $\phi_{J}'$ является истиной.

$(\Leftarrow)$

Пусть $\phi'_{J}$ является истиной, тогда
\[I_{J}(\pcpotype) = \overline{\pcpuword{j_1}} \ldots \overline{\pcpuword{j_r}}\pcpeconstr = \overline{\pcpvword{j_1}} \ldots \overline{\pcpvword{j_r}}\pcpeconstr,\] является интерпретацией для $\phi_{J}$:
\[
    \begin{array}{l}
    \interpretation{J}{\phi_{J}} = \\
    [0.1em] \\
    \pcpdword \subtype \interpretation{J}{\pcpotype} \land \pcpuword{J}\pcpeconstr \subtype \interpretation{J}{\pcpotype} \land \pcpvword{J}\pcpeconstr \subtype \interpretation{J}{\pcpotype} \land \interpretation{J}{\pcpotype} \notsubtype \pcpeconstr = \\
    [0.1em] \\
    \pcpdword \subtype \overline{\pcpuword{j_1}} \ldots \overline{\pcpuword{j_r}}\pcpeconstr \land \pcpuword{J}\pcpeconstr \subtype \overline{\pcpuword{j_1}} \ldots \overline{\pcpuword{j_r}}\pcpeconstr \land \\
    \pcpvword{J}\pcpeconstr \subtype \overline{\pcpvword{j_1}} \ldots \overline{\pcpvword{j_r}}\pcpeconstr \land \overline{\pcpuword{j_1}} \ldots \overline{\pcpuword{j_r}}\pcpeconstr \notsubtype \pcpeconstr \Leftrightarrow \\
    [0.1em] \\
    \overline{\pcpuword{j_1}} \ldots \overline{\pcpuword{j_r}}\pcpeconstr \subtype \overline{\pcpuword{j_1}} \ldots \overline{\pcpuword{j_r}}\pcpeconstr \land
    \overline{\pcpvword{j_1}} \ldots \overline{\pcpvword{j_r}}\pcpeconstr \subtype \overline{\pcpvword{j_1}} \ldots \overline{\pcpvword{j_r}}\pcpeconstr \land \\
    \overline{\pcpuword{j_1}} \ldots \overline{\pcpuword{j_r}}\pcpeconstr \notsubtype \pcpeconstr \Leftrightarrow \top
    \end{array}
\]
\qed\end{proof*}
\begin{thm}{\label{thm:pcpphi3}}
   Формула
\[
      \phi_{J} \eqdef \pcpdword \subtype \pcpotype \land \pcpuword{J}\pcpeconstr \subtype \pcpotype \land \pcpvword{J}\pcpeconstr \subtype \pcpotype \land \pcpotype \notsubtype \pcpeconstr
\]
выполнима тогда и только тогда, когда $J$ является решением \pcp{} с парами слов $\lbrace (\overline{\pcpuword{1}}, \overline{\pcpvword{1}}), \ldots , (\overline{\pcpuword{n}}, \overline{\pcpvword{n}})\rbrace$ над алфавитом $\setctor{\pcpfstsym{},\ldots,\pcpsndsym{}}$, то есть
\[
    \overline{\pcpuword{j_1}} \ldots \overline{\pcpuword{j_r}} = \overline{\pcpvword{j_1}} \ldots \overline{\pcpvword{j_r}}
\]
\end{thm}
\begin{proof*}
По~\autoref{lem:pcpphij} $\phi_{J}$ выполнима тогда и только тогда, когда $r>0$ и
\[
    \phi'_{J} \eqdef \overline{\pcpvword{j_1}}\ldots\overline{\pcpvword{j_r}}\pcpeconstr = \overline{\pcpuword{j_1}}\ldots\overline{\pcpuword{j_r}}\pcpeconstr
\] является истиной.

$(\Rightarrow)$
Пусть $\phi'_{J}$ является истиной. Тогда $J$ является такой непустой последовательностью индексов, что конкатенация слов $\overline{\pcpuword{j_1}}, \,\ldots,\,\overline{\pcpuword{j_r}}$ равно конкатенации слов $\overline{\pcpvword{j_1}}, \,\ldots,\,\overline{\pcpvword{j_r}}$. Следовательно $J$ является решением для \pcp.

$(\Leftarrow)$

Пусть $J$ является решением PCP. Это значит, что типы $\overline{\pcpvword{j_1}}\ldots\overline{\pcpvword{j_r}}\pcpeconstr$ и $\overline{\pcpuword{j_1}}\ldots\overline{\pcpuword{j_r}}\pcpeconstr$ равны. Кроме того, так как $J$ является решением \pcp, она не пуста. Из этого сделает истинность $\phi'_{J}$.
\qed\end{proof*}

\begin{thm}
Формула
\[
\psi \eqdef \pcpdword \subtype \pcpotype \land \pcpxtype \subtype \pcpotype \land \pcpytype \subtype \pcpotype \land \pcpotype \notsubtype \pcpeconstr \land \phi_{1} \land \phi_{2}
\]
выполнима тогда и только тогда, когда \pcp{} с парами слов $\lbrace (\overline{\pcpuword{1}}, \overline{\pcpvword{1}}), \ldots , (\overline{\pcpuword{n}}, \overline{\pcpvword{n}})\rbrace$ над алфавитом $\setctor{\pcpfstsym{},\ldots,\pcpsndsym{}}$ имеет решение.
\end{thm}
\begin{proof*}
По~\autoref{thm:pcpindices} формула $\phi_{1} \land \phi_{2}$ имеет множество решений с интерпретациями $\interpretation{J}{I}$ такими, что:
\[
    \begin{array}{c}
        \interpretation{J}{\pcpxtype} = \pcpuword{J}\pcpeconstr, \quad I_{J}(\pcpytype) = \pcpvword{J}\pcpeconstr, \quad \interpretation{J}{\pcpztype} = \pcpwword{J}\pcpeconstr, \\
        [0.6em] \\
        \interpretation{J}{\pcpptype} = \pcprconstr^{r}\pcpeconstr, \quad \interpretation{J}{\pcpqtype} = \pcpsconstr^{r}\pcpeconstr
        
    \end{array}
\]
Следовательно выполнимость формулы $\psi$ эквивалентна выполнимости
\[
    \psi' \eqdef \bigvee_{J \in \finiteseq{N}} \interpretation{J}{\psi} =
\]
\[
    \bigvee_{J \in \finiteseq{N}} \pcpdword \subtype \pcpotype \land \pcpuword{J}\pcpeconstr \subtype \pcpotype \land \pcpvword{J}\pcpeconstr \subtype \pcpotype \land \pcpotype \notsubtype \pcpeconstr =
    \bigvee_{J \in \finiteseq{N}} \phi_{J}
\]
По~\autoref{thm:pcpindices}, $\phi_{J}$ выполнима тогда и только тогда, когда $J$ является решением \pcp. Следовательно, $\psi'$ выполнима тогда и только тогда, когда существует последовательность индексов $J$ такая, что она является решением \pcp.
\qed\end{proof*}

\begin{crlr}
\subtypesat{} неразрешима.
\end{crlr}

\begin{crlr}
\subtypesat{} неразрешима даже для нерасширяющихся таблиц классов без контравариантных конструкторов.
\end{crlr}

\begin{crlr}
\subtypesat{} неразрешима даже для нерасширяющихся таблиц классов, содержащих только нульарные и унарные конструкторы.
\end{crlr}

\begin{crlr}
\subtypesat{} неразрешима даже для бескванторной конъюнкции атомов.
\end{crlr}

\begin{crlr}
\subtypesat{} неразрешима даже для бескванторной конъюнкции положительных атомов.
\end{crlr}
\begin{proof*}
Для доказательства мы просто должны исключить тривиальное решение, при котором все переменные отображаются в $\pcpeconstr{}$, без использования атома $\pcpotype \notsubtype \pcpeconstr$ в $\psi$.

Этого можно достигнуть, например, добавлением в \pcpctid{} новой записи
\begin{alignat*}{3}
    \classtableline{\pcpdword_0}{\overline{\pcpuword{1}}\pcpdword,\ldots, \overline{\pcpuword{n}}\pcpdword}
\end{alignat*}
и заменой $\psi$ на
\[
    \psi \eqdef \pcpdword_0 \subtype \pcpotype \land \pcpxtype \subtype \pcpotype \land \pcpytype \subtype \pcpotype \land \phi_{1} \land \phi_{2}
\]
\qed\end{proof*}
