\section{Обзор схожих работ}

\paragraph{Подтипирование между закрытыми типами}
Существует ряд из исследовательских работ о номинальном подтипировании с вариантностью между закрытыми типами. В одной из последних таких работ была показана Тьюринг-полнота подтипирования в \java{}~\cite{grigore2017java}. C++ шаблоны также Тьюринг-полны~\cite{veldhuizen2003c++}. Системы типов Scala \cite{odersky2016scaling}, OCaml \cite{lillibridge1997translucent, rossberg1999undecidability} и Haskell с расширениями~\cite{sulzmann2007understanding} неразрешимы. Напротив, подтипирование в \dotnet{} разрешимо~\cite{emir2006variance,kennedy2006decidability}. Упомянутые статьи формализуют и исследуют \emph{проверку типов} в некоторых языках программирования. Мы же исследуем подтипирование учитывая открытые типы, основываясь на результатах о подтипировании между закрытыми типами.

\paragraph{Выполнимость ограничений}
Одной из последних близких нам работ является~\cite{sherman2015deciding}. В ней, действуя в соответствии с  теми же целями, сводят проблему выполнимости для частично упорядоченных множеств, основанных на типах, к проблеме выполнимости в языке первого порядка, и предлагают использовать SMT-решатели для решения ограничений. В отличие от нашей работы, рассматривается только номинальный фрагмент системы типов \java{} без обобщённых типов.

Проблема выполнимости ограничений, и её вычислительная сложность для более общих (в сравнении с ~\cite{sherman2015deciding}) фрагментов системы типов исследуется в ~\cite{pratt1996satisfiability, frey1997satisfying, kuncak2003structural, niehren2005complexity}. Эти работы исследуют ограничения на конечные и рекурсивные типы, структурное и не структурное подтипрование и типовые конструкторы с ковариантными и контравариантными типовыми параметрами. Наша работа изучает более общую проблему.

В~\cite{su2002first} показывается неразрешимость задачи выполнимости ограничений для не структурного подтипирования. Этот результат влечёт неразрешимость \subtypesat{}, но для его доказательства используется значительно больший фрагмент логики первого порядка (например, формулы с кванторами всеобщности). Наше доказательство использует только бескванторные конъюнкции положительных атомов, однако, оно также использует особенности номинального подтипирования с вариантностью, которых нет в не структурном подтипировании.
