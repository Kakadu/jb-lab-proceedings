\section{Related works} \label{section_related}
Problems in many areas can be reduced to one of the formal-languages-constrained path problems~\cite{barrett2000formal}. For example, various problems of static code analysis~\cite{bastani2015specification,xu2009scaling} can be formulated in terms of the context-free language reachability~\cite{reps1998program} or in terms of the linear conjunctive language reachability~\cite{zhang2017context}. 

One of the well-known problems in the area of graph database analysis is the language-constrained path querying. For example, the regular language constrained path querying~\cite{reutter2017regular, fan2011adding, abiteboul1997regular, nole2016regular}, and the context-free language constrained path querying.

There are a number of solutions~\cite{hellingsRelational, GraphQueryWithEarley, RDF} for context-free path query evaluation w.r.t. the relational query semantics, which employ such parsing algorithms as CYK~\cite{kasami, younger} or Earley~\cite{Grune}. Other examples of path query semantics are single-path and \textit{all-path query semantics}. The all-path query semantics requires presenting all possible paths from node $m$ to node $n$ whose labeling is derived from a non-terminal $A$ for all triples $(A, m, n)$ evaluated using the relational query semantics. Hellings~\cite{hellingsPathQuerying} presented algorithms for the context-free path query evaluation using the single-path and the all-path query semantics. If a context-free path query w.r.t. the all-path query semantics is evaluated on cyclic graphs, then the query result can be an infinite set of paths. For this reason, in~\cite{hellingsPathQuerying}, annotated grammars are proposed as a possible solution.

In~\cite{GLL}, the algorithm for context-free path query evaluation w.r.t. the all-path query semantics is proposed. This algorithm is based on the generalized top-down parsing algorithm~---~GLL~\cite{scott2010gll}. This solution uses derivation trees for the result representation which is more native for grammar-based analysis. The algorithms in~\cite{GLL, hellingsPathQuerying} for the context-free path query evaluation w.r.t. the all-path query semantics can also be used for query evaluation using the relational and the single-path semantics.

Our work is inspired by Valiant~\cite{valiant}, who proposed an algorithm for general context-free recognition in less than cubic time. This algorithm computes the same parsing table as the CYK algorithm but does this by offloading the most intensive computations into calls to a Boolean matrix multiplication procedure. This approach not only provides an asymptotically more efficient algorithm but it also allows us to effectively apply GPGPU computing techniques. Valiant's algorithm computes the transitive closure $a^+$ of a square upper triangular matrix $a$. Valiant also showed that the matrix multiplication operation $(\times)$ is essentially the same as $|N|^2$ Boolean matrix multiplications, where $|N|$ is the number of non-terminals of the given context-free grammar in Chomsky normal form.

Hellings~\cite{hellingsRelational} presented an algorithm for the context-free path query evaluation using the relational query semantics. According to Hellings, for a given graph $D = (V, E)$ and a grammar $G = (N, \Sigma, P)$ the context-free path query evaluation w.r.t. the relational query semantics reduces to a calculation of the context-free relations $R_A$. Thus, in this work, we focus on the calculation of these context-free relations. Also, Hellings~\cite{hellingsRelational} presented an algorithm for the context-free path query evaluation using the single-path query semantics which evaluates paths of minimal length for all triples $(A,m,n)$, but also noted that the length of these paths is not necessarily upper bounded. Thus, in this work, we evaluate an arbitrary path for all triples $(A,m,n)$.

Yannakakis~\cite{transitive-closure} analyzed the reducibility of various path querying problems to the calculation of the transitive closure. He formulated a problem of Valiant's technique generalization to the context-free path query evaluation w.r.t. the relational query semantics. Also, he assumed that this technique cannot be generalized for arbitrary graphs, though it does for acyclic graphs.

Thus, the possibility of reducing the context-free path query evaluation using the relational and the single-path query semantics to the calculation of the transitive closure is an open problem.

Also, there is an algorithm~\cite{zhang2017context} for path querying with linear conjunctive grammars and relational query semantics. This grammars have no more than one nonterminal in each conjunct of the rule. Thus, the possibility of creating an algorithm for path query evaluation w.r.t. conjunctive grammars of an arbitrary form is an open problem.
